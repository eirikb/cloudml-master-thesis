\begin{figure}[tb]
  \begin{center}
    \subfigure[Application modules]{
      \begin{tikzpicture}[scale=0.8, transform shape]
        \node (AuxNode01) [text width=4cm] {};
        \node (Engine) [class, left=of AuxNode01, rectangle split, rectangle split parts=2] { 
          \textbf{Engine} 
          \nodepart{second}Entry point. Orchestration.
        };
        \node (AuxNode02) [left=of Engine] {};
        \node (Kernel) [class, below=of AuxNode01, rectangle split, rectangle split parts=2] { 
          \textbf{Kernel} 
          \nodepart{second}Node domains. Converts JSON to Node Entities.
        };
        \node (Repository) [class, above=of AuxNode01, rectangle split, rectangle split parts=2] { 
          \textbf{Repository} 
          \nodepart{second}Instance domains. Convert Nodes to Instances.
        };
        \node (Cloud-Connector) [class, right=of AuxNode01, rectangle split, rectangle split parts=2] { 
          \textbf{Cloud-Connector} 
          \nodepart{second}Connects to providers (jclouds).
        };

        \draw [arrow] (Engine) -- (Kernel);
        \draw [arrow] (Engine) -- (Cloud-Connector);
        \draw [arrow] (Engine) -- (Repository);
        \draw [arrow] (Cloud-Connector) -- (Kernel.east);
        \draw [arrow] (Cloud-Connector) -- (Repository.east);
        \draw [arrow] (Repository) -- (Kernel);
        \draw [extend] (AuxNode02) -- (Engine);
      \end{tikzpicture}
    }

    \subfigure[Legend]{
      \begin{tikzpicture}[scale=0.7, transform shape]
        \node (Module) [class, label=below:Application modules] { Module };

        \node (AuxNode01) [right=of Module] {};
        \node (AuxNode02) [right=of AuxNode01] {};
        \node (AuxNode03) [right=of AuxNode02] {};
        \node (AuxNode04) [right=of AuxNode03] {};

        \draw[arrow] (AuxNode01) -- node[below] {Dependency} (AuxNode02);
        \draw[extend] (AuxNode03) -- node[below] {Entry} (AuxNode04);
      \end{tikzpicture}
    }
  \end{center}
  \caption{Architecture of cloudml-engine}
  \label{fig:cloudml-engine}
\end{figure}
