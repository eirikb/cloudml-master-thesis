\begin{figure}[tb]
  %\hspace*{-2cm}
  \begin{sequencediagram}[scale=0.9, transform shape]
    \newthread{u}{:User}
    \newthreadhack{c}{:CloudML}
    \newinst{r1}{r1:RI}
    \newinst{r2}{r2:RI}
    \newinst{r3}{r3:RI}
    \newthread{a}{:AWS}
    
    \begin{call}{u}{build(\texttt{account},List(\texttt{template}))}{c}{
        List(\texttt{r1}, \texttt{r2}, \texttt{r3})}
      \begin{call}{c}{Initialize()}{r1}{}
      \end{call}
      \begin{call}{c}{Initialize()}{r2}{}
      \end{call}
      \begin{call}{c}{Initialize()}{r3}{}
      \end{call}
    \end{call}

    %\begin{messcall}{c}{provision(\texttt{r1})}{a}
    %\end{messcall}
    %\begin{messcall}{c}{provision(\texttt{r2})}{a}
    %\end{messcall}
    %\begin{messcall}{c}{provision(\texttt{r3})}{a}
    %\end{messcall}
    \begin{messcall}{c}{provision nodes(\texttt{r1}, \texttt{r2}, \texttt{r3})}{a}
    \end{messcall}
    
    \begin{call}{u}{getStatus()}{r1}{\emph{"Building"}}
    \end{call}
    \begin{call}{u}{getStatus()}{r2}{\emph{"Building"}}
    \end{call}
    
    \begin{messcall}{a}{status(\texttt{r1}, \emph{"Starting"})}{c}
    \end{messcall}
    \begin{messcall}{c}{update(\emph{"Starting"})}{r1}
    \end{messcall}
    
    
    \begin{call}{u}{getStatus()}{r1}{\emph{"Starting"}}
    \end{call}
    \begin{call}{u}{getStatus()}{r2}{\emph{"Building"}}
    \end{call}

    \begin{messcall}{a}{status(\texttt{r1}, \emph{"Started"})}{c}
    \end{messcall}
    \begin{messcall}{c}{update(\emph{"Started"})}{r1}
    \end{messcall}
    \begin{messcall}{a}{status(\texttt{r2}, \emph{"Starting"})}{c}
    \end{messcall}
    \begin{messcall}{c}{update(\emph{"Starting"})}{r2}
    \end{messcall}

    \begin{call}{u}{getStatus()}{r1}{\emph{"Started"}}
    \end{call}
    \begin{call}{u}{getStatus()}{r2}{\emph{"Starting"}}
    \end{call}
    \begin{call}{u}{getStatus()}{r3}{\emph{"Building"}}
    \end{call}
  \end{sequencediagram}
  
  \caption{Asynchronous provisionning with three nodes. \texttt{RI} is an abbrevation of RuntimeInstance, to save space.}
  \label{fig:sequence-threenodes}
\end{figure}
