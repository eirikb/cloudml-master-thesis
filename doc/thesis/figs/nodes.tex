\begin{figure}
  \tikzstyle{browser}=[rectangle, dashed, draw=black, rounded corners, fill=white!40, drop shadow,
  text centered, anchor=north, text=black, text width=3.5cm, top color=white, bottom color=black!10]

    \subfigure[Single node]{
      \begin{tikzpicture}[scale=0.7, transform shape]
        \node (Browser) [browser] { \textbf{Browser} };
        \node (Node) [class, right=of Browser] { \textbf{Front-end And Back-end} };

        \draw[extend] (Browser.east) -- (Node.west);
      \end{tikzpicture}
      \label{fig:singlenode}
    }

    \subfigure[Two nodes]{
      \begin{tikzpicture}[scale=0.7, transform shape]
        \node (Browser) [browser] { \textbf{Browser} };
        \node (Frontend) [class, right=of Browser] { \textbf{Front-end} };
        \node (Backend) [class, right=of Frontend] { \textbf{Back-end} };

        \draw[extend] (Browser.east) -- (Frontend.west);
        \draw[extend] (Frontend.east) -- (Backend.west);
      \end{tikzpicture}
      \label{fig:twonodes}
    }

    \subfigure[Three nodes]{
      \begin{tikzpicture}[scale=0.7, transform shape]
        \node (Browser) [browser] { \textbf{Browser} };
        \node (LoadBalancer) [tdiamond, right=of Browser] { \textbf{Load Balancer} };
        \node (Frontend1) [class, right=of LoadBalancer, yshift=1cm] { \textbf{Front-end} };
        \node (Frontend2) [class, below=of Frontend1] { \textbf{Front-end} };
        \node (Backend) [class, right=of Frontend1, yshift=-1cm] { \textbf{Back-end} };

        \draw[extend] (Browser.east) -- (LoadBalancer.west);
        \draw[extend] (LoadBalancer.east) -- (Frontend1.west);
        \draw[extend] (LoadBalancer.east) -- (Frontend2.west);
        \draw[extend] (Frontend1.east) -- (Backend.west);
        \draw[extend] (Frontend2.east) -- (Backend.west);
      \end{tikzpicture}
      \label{fig:threenodes}
    }

    \subfigure[Several front-ends]{
      \begin{tikzpicture}[scale=0.7, transform shape]
        \node (Browser) [browser] { \textbf{Browser} };
        \node (LoadBalancer) [tdiamond, right=of Browser] { \textbf{Load Balancer} };
        \node (Frontend1) [class, right=of LoadBalancer, yshift=1cm] { \textbf{Front-end} };
        \node (Frontend2) [class, below=of Frontend1, yshift=-1cm] { \textbf{Front-end} };
        \node (Backend) [class, right=of Frontend1, yshift=-1cm] { \textbf{Back-end} };

        \draw[extend] (Browser.east) -- (LoadBalancer.west);
        \draw[extend] (LoadBalancer.east) -- (Frontend1.west);
        \draw[extend] (LoadBalancer.east) -- (Frontend2.west);
        \draw[extend] (Frontend1.east) -- (Backend.west);
        \draw[extend] (Frontend2.east) -- (Backend.west);

        \draw[loosely dotted, line width=3pt] (Frontend1.south) -- (Frontend2.north);
      \end{tikzpicture}
      \label{fig:frontends}
    }

    \subfigure[Several front-ends and back-ends (slaves)]{
      \begin{tikzpicture}[scale=0.7, transform shape]
        \node (Browser) [browser] { \textbf{Browser} };
        \node (LoadBalancer) [tdiamond, right=of Browser] { \textbf{Load Balancer} };
        \node (Frontend1) [class, right=of LoadBalancer, yshift=1cm] { \textbf{Front-end} };
        \node (Frontend2) [class, below=of Frontend1, yshift=-1cm] { \textbf{Front-end} };
        \node (Backend) [class, right=of Frontend1, yshift=-1cm] { \textbf{Back-end master} };
        \node (Slave1) [class, right=of Backend, yshift=1cm] { \textbf{Salve} };
        \node (Slave2) [class, below=of Slave1, yshift=-1cm] { \textbf{Slave} };

        \draw[extend] (Browser.east) -- (LoadBalancer.west);
        \draw[extend] (LoadBalancer.east) -- (Frontend1.west);
        \draw[extend] (LoadBalancer.east) -- (Frontend2.west);
        \draw[extend] (Frontend1.east) -- (Backend.west);
        \draw[extend] (Frontend2.east) -- (Backend.west);
        \draw[extend] (Backend.east) -- (Slave1.west);
        \draw[extend] (Backend.east) -- (Slave2.west);

        \draw[loosely dotted, line width=3pt] (Frontend1.south) -- (Frontend2.north);
      \end{tikzpicture}
      \label{fig:frontendbackends}
    }

    \subfigure[Legend]{
      \begin{tikzpicture}[scale=0.7, transform shape]
        \node (Browser) [browser, label=below:Non-system interaction] { Browser };
        \node (Node) [class, right=of Browser, label=below:Provisioned instance] { Node };

        \node (AuxNode01) [right=of Node] {};
        \node (AuxNode02) [right=of AuxNode01] {};
        \node (AuxNode03) [right=of AuxNode02] {};
        \node (AuxNode04) [right=of AuxNode03] {};

        \draw[extend] (AuxNode01) -- node[below] {Connection flow} (AuxNode02);
        \draw[loosely dotted, line width=3pt] (AuxNode03) -- node[below] {n-times} (AuxNode04);
      \end{tikzpicture}
    }

  \caption{Different architectural ways to provision nodes (topologies).}
  \label{fig:BankManager}
\end{figure}
