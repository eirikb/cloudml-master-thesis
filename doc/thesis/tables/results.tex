\begin{table}
  \begin{center}
    \caption{Result of how requirements were tackled.}
    \begin{tabular}{| l | p{7cm} |}
      \hline
        \textbf{Requirement} &
        \textbf{CloudML solution} \\
      \hline
        \citereq{software-reuse} & The engine rely heavily on an external library,
                                   interfacing cloud providers with the engine.
                                   This library is used to interface cloud providers
                                   with the engine.  \\ \hline
        \citereq{foundation} & Implementation utilizing Scala,
                               a \emph{``state-of-the-art''} language.
                               The language also leverage support for
                               software industry through the \myac{JVM}.\\ \hline
        \citereq{mda} & In CloudML a model-based meta-model is designed.
                        The solution also let end users design templates with models. \\ \hline
        \citereq{lexical-template} & The engine is capable of parsing and interpreting 
                                     \myac{JSON}-based templates.
                                     Then it configure and provision instances based on these templates.  \\ \hline
        \citereq{m@rt} & Asynchronous behavior introduces through the actor model,
                                      which is built into Scala.
                                      The actor model is combined with observer pattern,
                                      enhancing the engines ability to do asynchronous callbacks. \\ \hline
        \citereq{multi-cloud} & The library \emph{jclouds} is included in the engine,
                                giving it support for $24$ providers. \\ \hline
    \end{tabular}
  \end{center}
  \label{table:results}
\end{table}

