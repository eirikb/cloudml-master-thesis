\mychapter{conclusions}{Conclusions}

In this thesis four main parts have been presented.
First the background part introducing the domain of cloud computing and model-driven engineering.
Then the second part highlighting sets of technologies, frameworks, ideas and 
\myac{API}s which are currently used in the two domains.
Third the challenges with these solutions are stressed, as well as a set of \emph{requirements}
CloudML must fulfill to tackle these challenges.
Lastly, CloudML is presented, in three phases,
\begin{ii}
  \iitem vision,
  \iitem design and
  \iitem implementation.
\end{ii}

In the vision chapter the core idea of CloudML were introduced,
and even means to tackle \citereq{m@rt} were outlined through pure vision.
The design chapter stated how CloudML should be built up,
what the meta-model should look like,
what underlying technologies should be used.
All through a scenario where Alice performs provisioning.
In this chapter the means to tackle \citereq{m@rt} are reinforced through
the view of design.
The requirements of \citereq{foundation} and \citereq{software-reuse}
are addressed through what underlying technology to use and alternatives.
Lastly the implementation chapter outline how CloudML is implemented
as \emph{cloudml-engine}, and how this solution is built up.
Both \citereq{mda} and \citereq{lexical-template} are tackled in this chapter
by concretely choosing data format and syntax based on the design chapter.


\section{Results}
\begin{table}
  \begin{tabular*}{\textwidth}{@{\extracolsep{\fill}}| l | l | l | l | l | l |}
      \hline
        \textbf{State of the art} & 
        \textbf{\citereq{software-reuse}} & 
        \textbf{\citereq{foundation}} & 
        \textbf{\citereq{mda}} & 
        \textbf{\citereq{m@rt}} & 
        \textbf{\citereq{lexical-template}} \\
      \hline
     Amazon CloudFormation & & & Yes & No & Yes \\ \hline
     CA Applogic & & & Yes & & No \\ \hline
     jclouds & & Yes & Partly & No & No \\ \hline
     mOSAIC & Yes & Yes & No & No & No \\ \hline
     Amazon Beanstalk & Yes & Yes & No & No & No \\ \hline
     CloudML & Yes & Yes & Yes & Yes & Yes \\ \hline
  \end{tabular*}
  \caption{Comparing selected elements from \citechap{state-of-the-art} with requirements.}
  \label{table:requirements-comparison}
\end{table}



The implementation is validated through an experiment where
it is physically executed against two providers, \myac{AWS} and Rackspace.
This experiment concludes the work put into CloudML at this point to be successful.

The requirements from \citechap{requirements} are compared against selected
technologies and frameworks from \citechap{state-of-the-art}.
These comparisons are expressed in \citetable{requirements-comparison}.
