\mychapter{vision}{Envision, concepts and principles}
%\begin{figure}
  \begin{center}
    \begin{tikzpicture}
      \stickman{s1}
    \end{tikzpicture}
  \end{center}
  \caption{Big picture}
  \label{fig:bigpicture}
\end{figure}


In this chapter the core approach and steps on the research to implementing CloudML will be described.
In the previous chapters, \citechap{challenges} and \citechap{requirements}, 
challenges and requirements were identified and described.
In this part of the thesis (\emph{contribution}) the challenges will be resolved
by applying requirements.

\section{Requirement dependencies}
\begin{figure}
  \begin{center}
    \subfigure[Requirement dependencies]{
      \begin{tikzpicture}[scale=0.8, transform shape]
        \node (SUT) [class] { 
          \textbf{Solid underlying technology} 
        };
        \node (LT) [class, below=of SUT] { 
          \textbf{Lexical template} 
        };
        \node (SR) [class, right=of SUT] { 
          \textbf{Software reuse} 
        };
        \node (M@RT) [class, above=of SUT] { 
          \textbf{Models@run.time} 
        };

        \draw [line] (SUT) -- (LT);
        \draw [line] (SUT) -- (SR);
        \draw [line] (SUT) -- (M@RT);
      \end{tikzpicture}
    }

    \subfigure[Legend]{
      \begin{tikzpicture}[scale=0.7, transform shape]
        \node (Requirement) [class] { Requirement };

        \node (AuxNode01) [text width=3cm, right=of Requirement] {};
        \node (AuxNode02) [right=of AuxNode01] {};

        \draw[line] (AuxNode01) -- node[below] {Dependency, both ways} (AuxNode02);
      \end{tikzpicture}
    }
  \end{center}
  \caption{Requirement dependencies}
  \label{fig:requirement-dependencies}
\end{figure}


Some of the requirements have depend on each other, for instance \emph{software reuse} is about finding
and utilizing an existing library or framework, but this will also directly affect or be affected by
programming language or application environment chosen in \emph{solid underlying technology} requirement.
There are three requirements, 
\begin{ii}
  \iitem Models@run.time,
  \iitem Software reuse and
  \iitem lexical template,
\end{ii}
where all have a two-way dependency to the \emph{solid underlying technology} requirement.
These dependencies are illustrated in \citefig{requirement-dependencies}.

\section{Requirements step by step}

In this section each of the requirements from \citechap{requirements} will be presented 
with a description overview of how to proceed to achieve them.

\paragraph{Solid underlying technologies.}
This will be the foundation of the solution, so finding good technologies has a high priority.
To find good technologies they have to be considered by several aspects, such as
\begin{ii}
  \iitem ease of use,
  \iitem community size,
  \iitem closed/open source,
  \iitem business viability,
  \iitem modernity and 
  \iitem matureness.
\end{ii}
Another important aspect is based on library or framework chosen for the \emph{software reuse} requirement,
as the library will directly affect some technologies such as programming language.
Different technologies have to be researched and to some degree physically tried out to identify
which aspects they fulfill.

Types of technologies are undefined but some are mandatory such as 
\begin{ii}
  \iitem programming language (Java, C\#) and 
  \iitem application environment (JDK, .NET).
\end{ii}
Beside this it is also important to state which type of application the solution should be, 
\begin{ii}
  \iitem GUI application,
  \iitem \myac{API} in form of public Web Service or
  \iitem \myac{API} in form of native library.
\end{ii}
The amount of different technologies is overwhelming so looking into all of them would be impossible,
therefore they must be narrowed down based on aspects such as popularity.

\paragraph{Software reuse.}
The \myac{API}s described in \citechap{state-of-the-art} are worthy candidates to meet this requirement.
Some research have already been done to indicate what their purpose is, 
how they operate and to some extent how to use them.
The approach here is to select libraries or framework that could match well with a chosen
\emph{solid underlying technology} or help fulfill this requirement.
Then the chosen \myac{API}s must be narrowed down to one single \myac{API} which will be
used in the solution application.

\paragraph{Model-driven approach.}
\ldots
\paragraph{Lexical templates.}
\paragraph{Models@run.time.}

\begin{center}
\line(1,0){250}
\end{center}

Battle plan, strategy?
How to attack the problem?
What is the problem, officer? -> Challenges, Requirements.

The core envision is to solve requirements from~\citechap{requirements} by applying a 
model-driven approach supported by modern technologies.
Main objective is to create a common model for nodes as a platform-independent 
model~\cite{agile:cuong10}
to justify \emph{multicloud} differences and 
at the same time base this on a human readable lexical format to resolve \emph{reproducibility} and
make it \emph{shareable}.

The concept and principle of CloudML is to be an easier and more reliable
path into cloud computing for IT-driven businesses of variable sizes.
the tool is envisioned to parse and execute template files representing topologies
of instances in the cloud. Targeted users are application developers without
cloud specific knowledge. The same files should be usable on other providers,
and alternating the next deployment stack should be effortless.
Instance types are selected based on properties within the template,
and additional resources are applied when necessary and available.
While the tool performs provisioning metadata of nodes is available.
In the event of a template being inconsistent with possibilities 
provided by a specific provider this error will be informed 
to the user and provision will halt.

There are many cloud providers on the global market today. These providers support many layers of cloud, 
such as PaaS (Platform as a Service) and IaaS (Infrastructure as a Service). 
This vast amount of providers and new technologies and services can be overwhelming for many companies and small and medium businesses. 
There are no practical introductions to possibilities and limitations to cloud computing, or the differences between different providers and services. 
Each provider has some kind of management console, usually in form of a web interface and API. 
But model driven approaches are inadequate in many of these environments. 
UML diagrams such as deployment diagram and component diagram are used in legacy systems to describe system architectures, 
but this advantage has yet to hit the mainstream of cloud computing management. 
It is also difficult to have co-operational interaction on a business level without using the advantage of graphical models.
The knowledge needed to handle one provider might differ to another, so a multicloud approach might be very resource-heavy on competence in companies. 
The types of deployment resources are different between the providers, even how to gain access to and handle running instances might be very different. 
Some larger cloud management application developers are not even providers themselves, but offer tooling for private cloud solutions.
Some of these providers have implemented different types of web based applications that let end users manage their cloud instances. 
The main problem with this is that there are no standards defining a cloud instance or links between instances and other services a provider offer.
If a provider does not offer any management interface and want to implement this as a new feature for customers, 
a standard format to set the foundation would help them achieve a better product for their end users.
These are some of the problems with cloud hosting today, and that CloudML will be designed to solve.
