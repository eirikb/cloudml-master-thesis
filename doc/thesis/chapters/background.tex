\chapter{Background}

\note{
  Explain some of the topics in my thesis. \\
  Here it is possible to introduce case study (BankManager) to ease writing
}

\todo {
  \begin{itemize}
    \item What is model-based engineering and benefits. \\
        Core concepts
  \end{itemize}
}


Cloud computing is gaining popularity and more companies are starting 
to explore the possibilities as well as the limitation to the cloud.
There are three main architectural service layers in cloud computing\cite{introduction:wozniak10}
namely \emph{Infrastructure-as-a-Service}~(IaaS), \emph{Platform-as-a-Service}~(PaaS)
and \emph{Software-as-a-Service}~(SaaS).
With IaaS on lowest vertical integration level closest to physical hardware and SaaS on the highest
level as runnable applications.
The main providers are Google, Amazon with \emph{Amazon Web Service}~(AWS)~\cite{aws} and Microsoft.
Some of providers are visualized in \citetable{providers}.

\begin{table}
  \begin{center}
    \begin{tabular}{ | l | l | l | }
      \hline
      \textbf{Provider} & \textbf{Service} & \textbf{Service Model} \\ \hline
      AWS & Elastic Compute Cloud & Infrastructure \\ \hline
      AWS & Elastic Beanstalk & Platform \\ \hline
      Google & Google App Engine & Platform \\ \hline
      CA & AppLogic & Infrastructure \\ \hline
      Microsoft & Azure & Platform and Infrastructure \\ \hline
      Heroku & Different services & Platform \\ \hline
      Nodejitsu & Node.js & Platform \\ \hline
      Rackspace & CloudServers & Infrastructure \\ \hline
    \end{tabular}
  \end{center}
  \caption{Common providers available services}
  \label{table:providerservices}
\end{table}



The nist definition of cloud computing~\cite{nist:mell11}
define \emph{Infrastrucutre-as-a-Service}~(IaaS) as
``The capability provided to the consumer is to provision 
processing, storage, networks, and other fundamental computing resources where the 
consumer is able to deploy and run arbitrary software, which can include operating 
systems and applications.'' 
And continue to state that `` The consumer does not manage or control the underlying cloud 
infrastructure but has control over operating systems, storage, deployed applications, and 
possibly limited control of select networking components (e.g., host firewalls).``

