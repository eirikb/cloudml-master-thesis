\mychapter{validation}{Validation \& Experiments}

To validate how CloudML addresses the requirements from~\citechap{requirements},
a topology of three nodes is provisioned~(\citefig{threenodes}).
This topology is the same as Alice used for her second scenario in \citesec{meta-model}.
The setup is sufficient to do a full deployment of the \emph{BankManager} application.

The implementation uses \myac{JSON} to define templates as a human readable serialization mechanism.
The lexical representation of \citefig{threenodes} can be seen in \citelist{validation-threenodes}. 
The whole text represents the \texttt{Template} of \citefig{threenodes} and consequently 
``nodes'' is a list of \texttt{Node} from the model.
The JSON is textual which makes it \emph{shareable} as files.
Once such a file is created it can be reused (\emph{reproducibility}) 
on any supported provider (\emph{multicloud}).
These benefits match the requirement \citereq{lexical-template}.

Characteristics of each node are carefully chosen based on each nodes feature area, for instance 
front-end nodes have more computation power, while the back-end node will have more disk.

\paragraph{Template.}
\begin{figure}
  \begin{center}
    \begin{minted}[mathescape,
                   linenos,
                   numbersep=5pt,
                   gobble=2,
                   frame=lines,
                   framesep=2mm]{json}
{
  "name": "test",
  "loadBalancer": {
    "name": "test",
    "protocol": "http",
    "loadBalancerPort": 80,
    "instancePort": 80
  },
  "nodes": [{
    "name": "test1"
  }, {
    "name": "test2"
  }]
}
    \end{minted}
  \end{center}
  \caption{Template for validation.}
  \label{list:validation-threenodes}
\end{figure}



Template!

\section{Comparisons}
\begin{table}
  \begin{tabular*}{\textwidth}{@{\extracolsep{\fill}}| l | l | l | l | l | l |}
      \hline
        \textbf{State of the art} & 
        \textbf{\citereq{software-reuse}} & 
        \textbf{\citereq{foundation}} & 
        \textbf{\citereq{mda}} & 
        \textbf{\citereq{m@rt}} & 
        \textbf{\citereq{lexical-template}} \\
      \hline
     Amazon CloudFormation & & & Yes & No & Yes \\ \hline
     CA Applogic & & & Yes & & No \\ \hline
     jclouds & & Yes & Partly & No & No \\ \hline
     mOSAIC & Yes & Yes & No & No & No \\ \hline
     Amazon Beanstalk & Yes & Yes & No & No & No \\ \hline
     CloudML & Yes & Yes & Yes & Yes & Yes \\ \hline
  \end{tabular*}
  \caption{Comparing selected elements from \citechap{state-of-the-art} with requirements.}
  \label{table:requirements-comparison}
\end{table}



Comparing challenges with some selected providers and technologies from \citechap{state-of-the-art}.
