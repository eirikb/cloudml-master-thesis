\mychapter{validation}{Validation \& Experiments}

To validate how CloudML addresses the requirements from~\citechap{requirements},
a topology of three nodes~(\citefig{threenodes}) is provisioned.
This topology is the same as Alice used for her second scenario in \citesec{meta-model}.
The setup is sufficient to do a full deployment of the \emph{BankManager} application.

The validation will provision on two different providers, \myac{AWS} and Rackspace.

\paragraph{Template.}
\begin{figure}
  \begin{center}
    \begin{minted}[mathescape,
                   linenos,
                   numbersep=5pt,
                   gobble=2,
                   frame=lines,
                   framesep=2mm]{json}
{
  "name": "test",
  "loadBalancer": {
    "name": "test",
    "protocol": "http",
    "loadBalancerPort": 80,
    "instancePort": 80
  },
  "nodes": [{
    "name": "test1"
  }, {
    "name": "test2"
  }]
}
    \end{minted}
  \end{center}
  \caption{Template for validation.}
  \label{list:validation-threenodes}
\end{figure}



The implementation uses \myac{JSON} to define templates as a human readable serialization mechanism.
The lexical representation of \citefig{threenodes} can be seen in \citelist{validation-threenodes}. 
There are a total of three files:
\begin{description}
  \item[account.json.]
    Used to authenticate against a provider.
    In \citelist{validation-threenodes} \texttt{aws-ec2} is set as \texttt{provider},
    \ie nodes are created on \myac{AWS}.
    The two other properties, \texttt{identity} and \texttt{credential} are used for authentication.
    For \myac{AWS} that means \emph{Access Key ID} and \emph{Secret Access Key},
    while for Rackspace this is \emph{username} and \emph{API Key}.
  \item[front-ends.json.]
    Defines front-end nodes of the topology.
    Each node have specific attributes regarding their tasks, similar to Alice's scenario,
    but as an addiontal precaution the \texttt{front-end} nodes have increased \myac{RAM}.
  \item[back-end.json.]
    Defines the back-end node of the topology.
    Even though this is a separate file it is a part of the same topology,
    and it is provisioned beside the front-end nodes.
\end{description}
%All nodes in the first template are bound to the load balancer~(\texttt{loadBalancer})
%defined within the template.
The topology is split into two templates to support a load balancer,
as every node within a template will be bound to a given load balancer.
This feature is implemented in \emph{cloudml-engine}, but at writing moment~(\date{April 2012}),
is not supported by jclouds.
%since the \texttt{back-end} node should not be bound to the load balancer.
The splitting is by design, as a \texttt{template} is not directly bound to a topology,
and is also why \texttt{build} accept a list of templates.
The whole text represents the \texttt{Template} and consequently 
``\texttt{nodes}'' is a list of \texttt{Node} from the model.
The JSON is textual which makes it \emph{shareable} as files.
Once such a file is created it can be reused (\emph{reproducibility}) 
on any supported provider (\emph{multicloud}).
These benefits match the requirement \citereq{lexical-template}.

\paragraph{Client.}
\begin{figure}[tb]
  \begin{center}
    \begin{minted}[mathescape,
                   linenos,
                   numbersep=5pt,
                   frame=lines,
                   framesep=2mm]{scala}

import no.sintef.cloudml.engine.Engine
import no.sintef.cloudml.repository.domain._
  ...
val runtimeInstances = Engine(account, templates)
println("Got " + runtimeInstances.size + " nodes")
runtimeInstances.foreach(ri => {
  println("Adding listener to: " + ri.instance.name + 
    " (" + ri.status + ")")
  ri.addListener( (event) =>  {
    event match {
      case Event.Property => 
      case Event.Status => 
        println("Status changed for " + ri.instance.name + 
          ": " + ri.stat
        if (ri.status ==  Status.Started) {
          println("Node " + ri.instance.name + 
            " is now running: " + ri)
        }
    }
  }
})
    \end{minted}
  \end{center}
  \caption{Code snipptes of client used for validation (Scala).}
  \label{list:validation-client}
\end{figure}



To validate the templates 

\paragraph{Building.}
\begin{figure}
  \begin{center}
    \begin{minted}[numbersep=5pt,
                   frame=lines,
                   framesep=2mm]{text}
mvn scala:run 
  -DaddArgs="account.json|front-ends.json|back-end.json"

[INFO] Scanning for projects...
[INFO]
   ...
Got 3 nodes
Adding listener to: front-end1 (Building)
Adding listener to: front-end2 (Configuring)
Adding listener to: back-end (Configuring)
Status changed for front-end1: Starting
Status changed for front-end1: Started
Node front-end1 is now running: 
  RuntimeInstance(Instance(front-end1,2,2,0,))
Status changed for front-end2: Building
Status changed for front-end2: Starting
Status changed for front-end2: Started
Node front-end2 is now running: 
  RuntimeInstance(Instance(front-end2,2,2,0,))
Status changed for back-end: Building
Status changed for back-end: Starting
Status changed for back-end: Started
Node back-end is now running: 
  RuntimeInstance(Instance(back-end,0,1,500,))
    \end{minted}
  \end{center}
  \caption{Output from running validation client.}
  \label{list:validation-output}
\end{figure}



\paragraph{After provisioning.}
\begin{figure}[tb]
  \includegraphics[width=\linewidth]{imgs/aws-console.png}
  \caption{Screenshot of \myac{AWS} console after validation provisioning.}
  \label{fig:validation-aws}
\end{figure}

\begin{figure}[tb]
  \includegraphics[width=\linewidth]{imgs/rackspace-console.png}
  \caption{Screenshot of Rackspace console after provisioning.}
  \label{fig:validation-rackspace}
\end{figure}


\section{Comparisons}
\begin{table}
  \begin{tabular*}{\textwidth}{@{\extracolsep{\fill}}| l | l | l | l | l | l |}
      \hline
        \textbf{State of the art} & 
        \textbf{\citereq{software-reuse}} & 
        \textbf{\citereq{foundation}} & 
        \textbf{\citereq{mda}} & 
        \textbf{\citereq{m@rt}} & 
        \textbf{\citereq{lexical-template}} \\
      \hline
     Amazon CloudFormation & & & Yes & No & Yes \\ \hline
     CA Applogic & & & Yes & & No \\ \hline
     jclouds & & Yes & Partly & No & No \\ \hline
     mOSAIC & Yes & Yes & No & No & No \\ \hline
     Amazon Beanstalk & Yes & Yes & No & No & No \\ \hline
     CloudML & Yes & Yes & Yes & Yes & Yes \\ \hline
  \end{tabular*}
  \caption{Comparing selected elements from \citechap{state-of-the-art} with requirements.}
  \label{table:requirements-comparison}
\end{table}



Comparing challenges with some selected providers and technologies from \citechap{state-of-the-art}.
