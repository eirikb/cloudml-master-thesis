\documentclass[UKenglish]{template/ifimaster}
%\usepackage[latin1]{inputenc}
\usepackage[T1]{fontenc,url}
\urlstyle{sf}
\usepackage{xargs}
\usepackage{babel}
\usepackage{textcomp}
\usepackage{template/uiosloforside}
\usepackage{varioref}
\usepackage{graphicx}
\usepackage{color}
\usepackage[usenames,dvipsnames]{xcolor}
\usepackage{tikz}
\usepackage{template/pgf-umlsd}
\usepackage{hyperref}
\usepackage{xspace}
\usepackage{subfigure}
\usepackage{minibox}
\usepackage{expl3}
\usepackage{times}
\usepackage{template/thesis}
\usepackage{template/mytikz}
\usepackage{minted}
\usepackage{framed}
\usepackage{latexsym}
\usepackage[numbers]{natbib}
\usepackage{setspace}
\usepackage{captcont}
\usepackage{pdfpages}
\usepackage{bibentry}
\usepackage[acronym,toc]{glossaries}
\usepackage{eurosym}
\usetikzlibrary{positioning,shapes,shadows,arrows,calc}

\makeglossaries
\myacdef{NIST}{National Institute of Standards and Technology}
\myacdef{AWS}{Amazon Web Service}
\myacdef{TPS}{Tweets Per Second}
\myacdef{SSH}{Secure Shell}
\myacdef{IaaS}{Infrastructure-as-a-Service}
\myacdef{PaaS}{Platform-as-a-Service}
\myacdef{SaaS}{Software-as-a-Service}
\myacdef{VPC}{Virtual Private Cloud}
\myacdef{VPS}{Virtual Private Servers}
\myacdef{EC2}{Elastic Compute Cloud}
\myacdef{GAE}{Google App Engine}
\myacdef{MDA}{Model-Driven Architecture}
\myacdef{CIM}{Computation Independent Model}
\myacdef{PIM}{Platform-Independent Model}
\myacdef{PSM}{Platform-Specific Model}
\myacdef{EBS}{Elastic Block Storage}
\myacdef{BSM}{Business Service Management}
\myacdef{JSON}{JavaScript Object Notation}
\myacdef{OMG}{Object Management Group}
\myacdef{JVM}{Java Virtual Machine}
\myacdef{ELB}{Elastic Load Balancer}
\myacdef{RDBMS}{Relation Database Management System}
\myacdef{MVC}{Model View Controller}
\myacdef{SOA}{Service-oriented Architecture}


\nobibliography*

\title{CloudML}
\subtitle{A DSL for model-based realization of applications in the cloud}
\author{Eirik Brandtz�g}
\date{Spring 2012}

\begin{document}
\uiosloforside[kind={Master thesis}]{}

\frontmatter{}
\maketitle{}
\onehalfspacing
%\doublespacing

%\chapter*{Abstract}
\begin{abstract}
  Cloud Computing offers a vast amount of resources,
  available for end users on a pay-as-you-go basis.
  The opportunity to choose between several cloud providers
  is alluded by complexity of cloud solution heterogeneity.
  Challenges with cloud deployment and resource provisioning
  are identified in this thesis through experiments,
  performing full cloud deployments with tools offered by cloud providers.
  To tackle these challenges a model-based language (named CloudML),
  is purposed to effectively tackle these challenges through abstraction.
  This language is supported by an engine able to provision nodes in \emph{``the cloud''}.
  Topologies for these nodes are defined through lexical templates.
  The engine supports $24$ cloud providers,
  through software reuse, by leveraging an existing library.
  Sufficient metadata for provisioned instances are provided through
  \emph{models@run.time} approach.
  The engine is implemented and experiments have been conducted towards
  provisioning on two of the major cloud providers, Amazon and Rackspace.
  These experiments are used to validate the effort of this thesis. 
  Perspectives and future visions of CloudML are expressed
  in the end of this thesis.
  This work is integrated as a part of the REMICS project.
\end{abstract}

\tableofcontents{}
\listoffigures{}
\listoftables{}

\chapter*{Preface}
%{\scriptsize
%\begin{minted}[]{brainfuck}
%>++++++++++[>+++++++>+++++++++++>+++++++++++>+++++++++++>++++++++++>+++++++++++
%+>++++++++++++>+++++++++++>++++++++++++>++++++++++>+++++>+++>+++++++>+++>++++++
%++++++>+++++++++++>++++++++++++>+++++++++++>++++++++++>+++>+++++++++++>++++++++
%+++>+++++++++++>++++++++++>+++>++++++++++++>+++++++++++>+++>++++++++++++>++++++
%++++>++++++++++>+++++++++++>+++++++++++>+++>++++++++>+++++++++++++++++++++++>++
%++++++++>++++++++++>++++++++++++>++++++++++++>+++++++++++>++++++++++>++++++++++
%+>+++>++++++++>+++++++++++>++++++++++++>++++++++++++>++++++++++>+++++++++++>+++
%>++++++++++>+++++++++++>+++++++++++>+++>++++++++++>+++++++++++>++++++++++++>+++
%>+++++++++++>+++++++++++>++++++++++++>++++++++++>+++++++++++>++++++++++>+++>+++
%+++++++>++++++++++>++++++++++>+++++++++++>++++++++++>++++++++++>++++++++++++>++
%+++++++++>+++++++++++>+++++++++++>+++>++++++++++++>+++++++++++>+++>++++++++++>+
%+++++++++++>++++++++++++>+++++++++++>++++++++++++>++++++++++++>+++>+++++++++++>
%++++++++++><<<<<<<<<<<<<<<<<<<<<<<<<<<<<<<<<<<<<<<<<<<<<<<<<<<<<<<<<<<<<<<<<<<<
%<<<<<<<<<<<<<<<<<<<<<<-]>+++>->++>++++>+>----->----->----->-->+>---->++>+++>++>
%->+>--->-->>++>-->----->--->+>++>---->+>++>---->++++>--->>--->++>+++>+++>-->---
%>----->---->----->+>>++>--->+>----->----->+>++++>++>++>+>++++>++>++++>----->---
%-->++>----->>----->--->>+>++>>+>>----->->--->---->----->+>>++>---->+>++>--->---
%-->----->----->----->---->++>->+><<<<<<<<<<<<<<<<<<<<<<<<<<<<<<<<<<<<<<<<<<<<<<
%<<<<<<<<<<<<<<<<<<<<<<<<<<<<<<<<<<<<<<<<<<<<>.>.>.>.>.>.>.>.>.>.>.>.>.>.>.>.>.>
%.>.>.>.>.>.>.>.>.>.>.>.>.>.>.>.>.>.>.>.>.>.>.>.>.>.>.>.>.>.>.>.>.>.>.>.>.>.>.>.
%>.>.>.>.>.>.>.>.>.>.>.>.>.>.>.>.>.>.>.>.>.>.>.>.>.>.>.>.>.>.>.>.
%\end{minted}
%}

I would like to thank SINTEF, and my supervisor Arne J{\o}rgen Berre,
for providing me with this opportunity to dive into the world of Cloud Computing.
I also want to thank S{\'e}bastien Mosser for his immense amount of 
assistance through designing and implementing of CloudML.
And of course for his accommodation through the writing process of this thesis,
including repeated feedback.
The Scala community, Specs2 mailing list and \emph{jclouds} IRC channel have been of great assistance,
guiding and helping through the implementation of CloudML.

\mainmatter{}

\mychapter{introduction}{Introduction}
\note{Short and sharp}

\todo{
  \begin{itemize}
    \item Main introduction
    \item Write lastly
  \end{itemize}
}


\part{Context}

\mychapter{background}{Cloud computing and model-driven artchitecture}
\begin{table}
  \begin{center}
    \begin{tabular}{ | l | l | l | }
      \hline
      \textbf{Provider} & \textbf{Service} & \textbf{Service Model} \\ \hline
      AWS & Elastic Compute Cloud & Infrastructure \\ \hline
      AWS & Elastic Beanstalk & Platform \\ \hline
      Google & Google App Engine & Platform \\ \hline
      CA & AppLogic & Infrastructure \\ \hline
      Microsoft & Azure & Platform and Infrastructure \\ \hline
      Heroku & Different services & Platform \\ \hline
      Nodejitsu & Node.js & Platform \\ \hline
      Rackspace & CloudServers & Infrastructure \\ \hline
    \end{tabular}
  \end{center}
  \caption{Common providers available services}
  \label{table:providerservices}
\end{table}



\note{
  Explain some of the topics in my thesis. \\
  Here it is possible to introduce case study (BankManager) to ease writing
}

\todo {
  \begin{itemize}
    \item What is model-based engineering and benefits. \\
        Core concepts
  \end{itemize}
}

\section{Cloud computing}

Cloud computing is gaining popularity and more companies are starting 
to explore the possibilities as well as the limitation to the cloud.
There are three main architectural service models in cloud computing\cite{nist:mell11}

namely \emph{Infrastructure-as-a-Service}~(IaaS), \emph{Platform-as-a-Service}~(PaaS)
and \emph{Software-as-a-Service}~(SaaS).
IaaS is on the lowest vertical integration level closest to physical hardware and SaaS on the highest
level as runnable applications.
\begin{itemize}
  \item \emph{IaaS}: 
    The main providers are Google, Amazon with \emph{Amazon Web Service}~(AWS)~\cite{aws} and Microsoft.
    Some of providers are visualized in \citetable{providerservices}.
    The NIST Definition of Cloud Computing~\cite{nist:mell11} define IaaS as
    \epigraph{The capability provided to the consumer is to provision 
      processing, storage, networks, and other fundamental computing resources where the 
      consumer is able to deploy and run arbitrary software, which can include operating 
      systems and applications.}{NIST, 2011}
    These are capabilities found in cloud provider services, 
    such as AWS \emph{Elastic Compute Cloud}~(EC2) and Rackspace CloudServers.
    NIST continue to state that 
    \epigraph{The consumer does not manage or control the underlying cloud 
      infrastructure but has control over operating systems, storage, deployed applications, and 
      possibly limited control of select networking components (\eg, host firewalls).}{NIST, 2011}
    According to Katarina Stanoevska-Slabeva~\cite{introduction:wozniak10} have 
    \emph{''infrastructure had been available as a service for quite some time``} and this 
    \emph{''has been referred to as utility computing``}, such as Sun Grid Compute Utility.
    IaaS is the most relevant cloud architecture layer for this paper, with focus
    on cloud provisioning.
  \item \emph{PaaS}:
    The PaaS model is defined as an capability consumers use to deploy onto cloud infrastructure.
    Deploying application that providers fully or partially support. For this kind of deployment
    consumers do not have to manage or control underlying infrastructure capabilities,
    and in some cases not even configuration.
    Examples of PaaS providers are Google with \emph{Google App Engine}~(GAE) and
    the company Heroku with their service with the same name.
    Multiple PaaS providers utilize EC2 as underlying infrastructure, examples of such
    providers are Heroku Nodester and Nodejitsu, this is a tendency with increasing popularity.
  \item \emph{SaaS}:
    SaaS has less relevance to this paper, nevertheless the core purpose
    is to provide whole web applications as services, in many cases end products.
    Google products such as gmail, Google Apps and  Google Calendar are examples of 
    SaaS applications.
\end{itemize}

Some of the most essential characteristics of cloud computing~\cite{nist:mell11} are:
\begin{itemize}
  \item \emph{On-demand self-service}: Consumers can do provisioning without any human interaction
  \item \emph{Broad network access}: Capabilities available over standard network mechanisms
  \item \emph{Resource pooling}: Physical and virtual resources are dynamically assigned
    and reassigned according to consumer demand
  \item \emph{Rapid elasticity}: Automatic capability scaling
  \item \emph{Measured service}: Monitoring and control of resource usages
\end{itemize}

There are four different deployment models according to The 
NIST Definition of Cloud Computing~\cite{nist:mell11}:
\begin{itemize}
  \item \emph{Private cloud}: Similar to classical infrastructures where hardware and
    drifting is owned and controlled by organizations themselves.
  \item \emph{Community cloud}: When several organizations share the same aspects of
    a private cloud (such as security requirements, policies, and compliance considerations),
    and therefore share infrastructure.
  \item \emph{Public cloud}: Infrastructure is open to the public.
    Cloud providers own the hardware and rent out IaaS and PaaS solutions to users.
    Examples of such providers are Amazon with AWS and Google with GAE.
  \item \emph{Hybrid cloud}: Combining private cloud or community cloud with public cloud.
    One benefit is to distinguish data from logic for purposes such as security issues,
    by storing sensitive information in a private cloud while computing with public cloud.
\end{itemize}

Of these deployment models \emph{public cloud} is the most relevant model for this paper
since the main purpose is to provision on public providers such as Amazon and Rackspace.
Beside these models defined by NIST there is another arising model known as 
\emph{virtual private cloud}, which is similar to \emph{public cloud} 
but with some security implications such as sandboxed network.

\section{Model-driven architecture approach}

By combining the world of cloud computing with the one of modeling 
it is possible to achieve benefits such as improved communication when designing 
a system and better understanding of the system itself.
This statement is emphasized by Booch (with co-authors) in one of his studies:
\epigraph{
  ``Modeling is a central
  part of all the activities that lead up to the deployment of good
  software. We build models to communicate the desired structure and
  behavior of our system. We build models to visualize and control the
  system's architecture. We build models to better understand the
  system we are building, often exposing opportunities for
  simplification and reuse. We build models to manage risk.''
}{Booch, 2005}
When it comes to cloud computing these definitions are even more important
because of financial aspects since provisioned nodes instantly starts to draw credit.
The definition of ``modeling'' can be assessed from the previous epigraph, but it is 
also important to choose correct models for the task.
Katarina Stanoevska-Slabeva states in one of her studies that grid computing
``\emph{is the starting point and basis for Cloud Computing.}''~\cite{introduction:wozniak10}.
As grid computing bear similarities towards cloud computing in terms of vitalization and utility computing
it is possible to use the same UML diagrams for IaaS as previously used in grid computing.
The importance of this re-usability of models is based on the origination of grid computing, \emph{eScience},
and the popularity of modeling in this research area.
The importance of choosing correct models is emphasized by Booch~\cite{unified:booch05}:
\epigraph{
  \begin{ii}\iitem The choice
  of what models to create has a profound influence on how a problem
  is attacked and how a solution is shaped. \iitem Every model may be
  expressed at different levels of precision. \iitem The best models
  are connected to reality. \iitem No single model is
  sufficient. Every nontrivial system is best approached through a
  small set of nearly independent models.\end{ii}
}{Booch, 2005}

\mychapter{state-of-the-art}{State of the Art in Provisioning}

There already exists scientific research projects, APIs, frameworks 
and other technologies which aim at consolidating, interfacing and utilizing cloud technologies.
This chapter introduces some of these concepts, and does this by dividing the chapter into four parts.
\begin{ii}
  \iitem Model-Driven Approaches which aims presenting frameworks and projects that utilize
  models on a larger scale.
  \iitem APIs are about frameworks that connects to cloud providers, often with multicloud support,
  these projects can be used as middleware for other systems.
  \iitem Deployments are about projects that do full deployment of applications, inclusing provisioning.
  These are often more academic.
  Lastly the chapter will discuss \iitem Example of cloud surveys, some real-world examples
  that might not be directly related to provisioning but are important and interesting regardless.
\end{ii}

\section{Model-Driven Approaches}

The following technologies are in some way model based.
That means they either use concrete diagrams or any other means of
modeling that can relate to Model-Driven Engineering.

\paragraph{Amazon AWS CloudFormation.}~\cite{aws}

\begin{figure}[tb]
  \begin{center}
    \begin{minted}[mathescape,
                   linenos,
                   numbersep=5pt,
                   gobble=2,
                   frame=lines,
                   framesep=2mm]{json}
{
  "Description": "Create an EC2 instance",
  "Parameters": {
    "KeyPair": {
      "Description": "For SSH access",
      "Type": "String"
    }
  },
  "Resources": {
    "Ec2Instance": {
      "Type": "AWS::EC2::Instance",
      "Properties": {
        "KeyName": { "Ref": "KeyPair" },
        "ImageId": "ami-1234abcd" 
      }
    }
  },
  "Outputs" : {
    "InstanceId": {
     "Description": "Instace ID of created instance",
     "Value": { "Ref": "Ec2Instance" }
    }
  },
  "AWSTemplateFormatVersion": "2010-09-09"
}
    \end{minted}
  \end{center}
  \caption{AWS CloudFormation template.}
  \label{fig:cloudformation-template}
\end{figure}



This is a service provided by Amazon from their popular \myac{AWS}.
It give users the ability to create template files in form of 
\myac{JSON} as seen in ~\citefig{cloudformation-template}, 
which they can load into AWS to create stacks of resources. 
A \emph{stack} is a defined set of resources in different amount and sizes, 
such as numerous instances,
one or more databases and a load balancer, although what types and sizes of resources is ambiguous.
To provision a stack with CloudFormation the template file (in JSON format) is first uploaded to
AWS which makes it accessible from AWS Management Console.

The template consist of three main sections, 
\begin{ii}\iitem \emph{Parameters}, \iitem \emph{Resources} and \iitem \emph{Outputs}.\end{ii}
The \emph{Parameters} section makes it possible to send parameters into the template, 
with this the template becomes a macro language by replacing 
references in the \emph{Resources} section with inputs from users. 
Input to the parameters are given inside the management console when 
provisioning a stack with a given template.
The \emph{Resource} section define types of resources that should be provisioned, the \emph{Type}
property is based on a set of predefined resource types such as \emph{AWS::EC2::Instance}
in Java package style.
The last section, \emph{Output}, will generate output to users when provisioning is complete,
here it is possible for users to pick from a set of variables to get the information they need.

This template system makes it easier for users to duplicate a setup many times, 
and as the templates support parameters this process can be as dynamic as the user design it to be. 
This is a model in form or lexical syntax, both the template itself 
and the resources that can be used.
For a company that is fresh in the world of cloud computing this service 
could be considered too advance. 
This is mainly meant for users that want to replicate a certain stack, 
with the ability to provide custom parameters. 
Once a stack is deployed it is only maintainable through the AWS Management Console, 
and not through template files. 
The format that Amazon uses for the templates is a good format, 
the syntax is in form of JSON which is readable and easy to use, 
but the structure and semantics of the template itself is not used by any 
other providers or cloud management tooling, 
so it can not be considered a multicloud solution. 
Even though JSON is a readable format, 
does not make it viable as a presentation medium on a business level.

\paragraph{CA Applogic.}~\cite{applogic}

\begin{figure}
  \begin{center}
  \end{center}
  \caption{CA Applogic}
    \includegraphics[width=\linewidth]{img/applogic.jpg}
  \label{fig:applogic}
\end{figure}


The Applogic platform is designed to manage CAs private cloud 
infrastructure~\cite{introducing-cloud-services}.
It also has a web based interface which let users manage their cloud resources 
as shown in~\citefig{applogic} which use and benefit from a model based approach.
It is based on graphical models which support interactive ``drag and drop'' functionalities.
This interface let users configure their deployments through a diagram with familiarities to 
UML component diagrams with interfaces and assembly connectors. 
They let users configure a selection of third party applications, 
such as Apache and MySQL, as well as network security, instances and monitoring. 
What CA has created is both an easy way into the cloud and it utilizes 
the advantages of model realizations. 
Their solution will also prove beneficial when conducting business level consulting
as it visualizes the structural layout of an application.
But this solution is only made for private clouds running their own controller, 
this can prove troublesome for migration, both in to and out of the infrastructure.

\paragraph{Madeira Cloud.}~\cite{madeiracloud}

\begin{figure}[tb]
  \includegraphics[width=\linewidth]{imgs/madeira.png}
  \caption{Madeira Cloud screenshot.}
  \label{fig:madiera}
\end{figure}


Madeira have created a tool which is similar to CA Applogic, but instead of focusing
on a private cloud solution they have created a tool specifically for \myac{AWS} \myac{EC2}.
Users can create \emph{stacks} with the available services in \myac{AWS} through 
dynamic diagrams.
These \emph{stacks} are live representations of the architecture and can be found 
and managed in the \myac{AWS} console as other \myac{AWS} services.
They also support storing running systems into template files which can be used
to redeploy identical copies and it should also handle configuration conflicts.
For identifying servers they use hostnames, which are bound to specific instances 
so end users don't have to bother with IP addresses.

\section{APIs}

Extensive work have been done towards simplifying and combining cloud technologies through
abstractions, interfaces and integrations.
Much of this work is in form of APIs, mostly in two different forms.
Either as programming libraries that can be utilized directly from a given 
programming language or environment such as Java or Python.
The other common solution is to have an online facade against public providers,
in this solution the APIs are mostly in \myac{REST} form.
\myac{REST}~\cite{rest:fielding00} is a software architecture for management of web resources
on a service. It uses the HTTP protocol and consecutive methods such as GET, POST, PUT and DELETE
to do tasks such as retrieve lists, items and create items.
APIs can be considered modeling approaches based on the fact they have a topology 
and hierarchical structure, 
but it is not a distinct modeling. 
A modeling language could overlay the code and help providing a clear overview, 
but the language directly would not provide a good overview of deployment. 
And links between resources can be hard to see, 
as the API lacks correlation between resources and method calls. 

\begin{figure}[tb]

  \begin{tikzpicture}[scale=1, transform shape]
    \node (Framework) [box, minimum width=8cm, minimum height=4.5cm] { };

    \node (Interface) [class, text width=2cm, yshift=0.5cm, xshift=-2.5cm] { 
      \textbf{Common  \\ interface} 
    };

    \node (Driver-EC2) [class, right=of Interface] { 
      \textbf{Driver - EC2} 
    };
    \node (Driver-Rackspace) [class, above=of Driver-EC2, yshift=-.5cm] { 
      \textbf{Driver - Rackspace} 
    };
    \node (Driver-Azure) [class, below=of Driver-EC2] { 
      \textbf{Driver - Azure} 
    };

    \node (EC2) [tcloud, xshift=1cm, right=of Driver-EC2] { \textbf{EC2} };
    \node (Rackspace) [tcloud, xshift=1cm, right=of Driver-Rackspace] { \textbf{Rackspace} };
    \node (Azure) [tcloud, xshift=1cm, right=of Driver-Azure] { \textbf{Azure} };

    \draw[arrow] (Interface) -- (Driver-EC2.west);
    \draw[arrow] (Interface) -- (Driver-Rackspace.west);
    \draw[arrow] (Interface) -- (Driver-Azure.west);

    \draw[arrow] (Driver-EC2) -- (EC2);
    \draw[arrow] (Driver-Rackspace) -- (Rackspace);
    \draw[arrow] (Driver-Azure) -- (Azure);
  \end{tikzpicture}

  \caption{Cloud drivers}
  \label{fig:drivers}
\end{figure}


\infobox{
  \subparagraph{Driver.}
  any of the API solutions use the term ``\emph{driver}'', it represents
  a module or component that fit into existing software and extend the support
  of external connections without changing the interface.
  A cloud \emph{driver} connects a given software to an existing cloud provider
  through this providers web based API (\myac{REST}), illustrated in \citefig{drivers}.
}

\paragraph{jclouds.}~\cite{jclouds}

This is a library written in Java and can be used from any \myac{JVM}-based language.
Provider support is implemented in \emph{drivers}, and they even support deployments
to some \myac{PaaS} solutions such as \myac{GAE}.
\emph{jclouds} divide their library in two different parts, one for computing powers 
such as \myac{EC2} and one for blob storage like S3. 
Some blob storage services are accessible on the compute side of the library such
as \myac{EBS}.
They support ``dry runs'' so a stack can be deployed as a simulation, not 
actually deploying it to a public cloud.
This is beneficial for testing deployments, and writing unit tests without initializing
connections, the library enhance this by providing a stub aimed at testing.

\paragraph{libcloud.}~\cite{libcloud}

Libcloud is an API that aims to support the largest cloud providers through a common API. 
The classes are based around \emph{drivers} that extends from a common ontology, 
then provider-specific attributes and logic is added to the implementation.
Libcoud is very similar to jclouds but the API code base is written in Python. 
The API is Python-only and could therefor be considered to have high tool-chain dependency.

\paragraph{Deltacloud.}~\cite{deltacloud}

Deltacloud has a similar procedure as jclouds and libcloud, but with a \myac{REST} API. 
So they also work on the term \emph{driver}, but instead of having a library to a 
programming language the users are presented with an web-based API they can call
on Deltacloud servers. 
As well as having similar problems as other APIs this approach means 
that every call has to go through their servers, similar to a proxy. 
This can work with the benefits that many middleware softwares have, such as caching, queues, 
redundancy and transformations.
The main disadvantages are single point of failure and version inconsistencies.
Deltacloud provide two sets of native libraries, one in Ruby and another in C, which
makes it easier to communicate with the \myac{REST} API.
Previously discussed \emph{jclouds} also support Deltacloud, as it would interface this
with a \emph{driver} as any other web-based API.

\section{Deployments}

There are also some solutions that specifically aim at full deployments,
contra provisioning single instances or services these solutions
provision everything needed to fully deploy an application with a given ontology.

\paragraph{mOSAIC.}~\cite{portable:petcu12} 

Aims at not only provisioning in the cloud, but deployment as well.
They focus on abstractions for application developers and state they can easily enable users to
\emph{``obtain the desired application characteristics (like
scalability, fault-tolerance, QoS, \etc.)''}~\cite{architecturing:petcu11}.
There are two abstraction layers, one for cloud provisioning 
and one for application-logic.
mOSAIC will select a proper cloud based on how developers describe their application,
several clouds can be selected based on their properties.
mOSAIC will use the \myac{IaaS} solutions of cloud providers to deploy users application,
then communication between these clouds will be done using 
``cloud based message queues technologies''.

\paragraph{RESERVOIR.}~\cite{reservoir:rochweger09}

\myac{RESERVOIR} is a European Union FP7 project, 
aiming at \emph{cloud federation} between private and 
hybrid clouds. With this a deployed application can distribute workload 
seamlessly between private and public clouds based on the applications requirements.
This is done by creating \emph{Reservoir sites}, one for each provider.
Each site is independent and run a \myac{VEE} which is managed by a \myac{VEEM}. 
The \myac{VEEM} communicate with other \myac{VEEM} and are able to do
federation between clouds. Each site must have the Reservoir software components 
installed, which makes it self-maintainable and self-monitoring.

\paragraph{Vega.}~\cite{simplifying:chieu10} 

Vega framework is a deployment framework aiming 
at full cloud deployments of multi-tier topologies, 
they also follow a model-based approach. 
The description of a given topology is done by using \myac{XML} files, with these files
developers can replicate a \emph{stack}.
The \myac{XML} contain information about the instances, such as \emph{ostype} for Operating System
and \emph{image-description} to describe properties of an instance such as amount of 
memory (\emph{req\_memory}).
They also allow small scripts to be written directly into the \myac{XML} through a node
\emph{runoncescript} which can do some additional configuration on a propagated node.
A resource manager keep track of resources in a system, grouping instances after their attributes.

\section{Example of cloud surveys}

In this section real-world examples are presented, such as popular IaaS and PaaS solutions
and other technologies which are widely used today.
Some solutions bear strong similarities to others, such as \myac{EC2} and Rackspace cloudservers,
for these only one solution will be discussed.

\paragraph{EC2.}

A central part of \myac{AWS}, it was Amazons initial step into the world of cloud computing when
they released \myac{EC2} as a service as public beta in $2006$.
This service is known to be the most basic service that cloud providers offer and is 
what makes Amazon an \myac{IaaS} provider.
When users rent \myac{EC2} services they are actually renting \myac{VPS} 
instances virtualized by Xen.
Although the instance itself can be considered a \myac{VPS} there are other factors 
that define it as a cloud service.
For instance cost calculations, monitoring and tightly coupled services surrounding \myac{EC2}.
Examples of these services are \myac{EBS} for block storage, 
\myac{ELB} for load balancing and \emph{Elastic IP} 
for dynamically assigning static IP addresses to instances.

Some of \myac{AWS} other services rely on \myac{EC2} such as \myac{AWS} Elastic Beanstalk 
and \emph{Elastic MapReduce}.
When purchasing these services it is possible to manage the \myac{EC2} instances 
through the \myac{EC2}-tab in \myac{AWS} console, but this is not mandatory
as they will be automatically managed through the original purchased service.
As mentioned earlier other \myac{PaaS} solutions delivered by independent companies
are using \myac{EC2} or other \myac{AWS} solutions.
Examples of these are Heroku, Nodester, DotCloud and Engine Yard which uses \myac{EC2}.
Example of companies using other \myac{AWS} services is Dropbox which uses S3j

\myac{EC2} is similar to services offered by other providers, 
such as Rackspace cloudservers, GoGrid cloud servers
and Linode Cloud.
Some of the additional services such as \myac{ELB} can also be found in other providers
which also offer these capabilities as services.

\paragraph{Amazon Beanstalk.}

Amazon has been known for providing IaaS solutions (\myac{EC2}) and services complementing 
either their IaaS or the \emph{Storage as a Service} solution S3.
Unlike some providers such as Microsoft and Google they had yet to introduce
a PaaS based solution, until they created Beanstalk.
This is the Amazon answer to PaaS, it is based on pre-configuring 
a stack of existing services such as \myac{EC2} for computing, 
\myac{EBS} for storage and \myac{ELB} for load balancing.
At the writing moment they support Java with Tomcat and PHP deployments.
The Java solution is based on uploading war-files to Beanstalk, then 
the service will handle the rest of the deployment.
For PHP the deployment is based on Git repositories, when pushing
code to a given repository Beanstalk will automatically deploy the new code
to an Apache httpd instance.


\mychapter{challenges}{Challenges in the cloud}

As cloud computing is growing in popularity it is also growing in complexity.
More and more providers are entering the market and different types of solutions are made.
There are few physical restrictions on how a provider should let their users do provisioning,
and little limitations in technological solutions.  
The result can be a complex and struggling introduction to cloud computing for users,
and provisioning procedure can alternate between providers.

This chapter will outline research on which has been conducted by
physical provisioning of an example application.
First the scenario will be introduced, describing the example application
and different means of provisioning in form of topologies.
Then challenges identified from the research will be presented.

\section{Scenario}

The following scenario was chosen because of how much it resembles actual solutions
used in industry today.
It uses a featureless example application meant to fit into scenario topologies
without having too much complexity.
Challenges should not be affected from errors or problems with the example application.
The application will be provisioned to a defined set of providers with a defined set of different topologies.

\paragraph{BankManager.}

To recognize challenges when doing cloud provisioning an example application~\cite{BankManager} was utilized.
The application (from here known as \emph{BankManager}) is a prototypical bank manager system
which support creating users and bank accounts and moving money between bank accounts and users.
The application is based on a three-tier architecture with 
\begin{ii} 
  \iitem presentation tier with a web-based interface,
  \iitem logic tier with controllers and services and
  \iitem database tier with models and entities.
\end{ii}
Three or more tiers in a web application is a common solution, even more so for applications 
based on the \myac{MVC} architectural pattern.
The advantage with this architecture is that the lowest tier (database) can be physically
detached from the tiers above, the application can then be distributed between several nodes.
It is also possible to have more tiers, for instance by adding a \emph{service} 
layer to handle re-usable logic.
Having more tiers and distributing these over several nodes is an architecture often
found in \myac{SOA} solutions.

\paragraph{Topologies.}

\begin{figure}
  \setlength{\unitlength}{1mm}

  \subfigure[Single node]{
    \begin{picture}(57.5, 17)
      \put(0, 5){\dashbox(15, 5){Browser}}
      \put(20, 2.5){\framebox(37.5, 12)}
      \put(22.5, 5){\dashbox(15, 5){Front-end}}
      \put(40, 5){\dashbox(15, 5){Back-end}}

      \put(15, 7.5){\vector(1, 0){5}}
      \put(22, 12){Node}
    \end{picture}
    \label{fig:singlenode}
  }

  \subfigure[Two nodes]{
    \begin{picture}(55, 5)
      \put(0, 0){\dashbox(15, 5){Browser}}
      \put(20, 0){\framebox(15, 5){Front-end}}
      \put(40, 0){\framebox(15, 5){Back-end}}

      \put(15, 2.5){\vector(1, 0){5}}
      \put(35, 2.5){\vector(1, 0){5}}
    \end{picture}
    \label{fig:twonodes}
  }

  \subfigure[Three nodes]{
    \begin{picture}(55, 15)
      \put(0, 5){\dashbox(15, 5){Browser}}
      \put(20, 0){\framebox(15, 5){Front-end}}
      \put(20, 10){\framebox(15, 5){Front-end}}
      \put(40, 5){\framebox(15, 5){Back-end}}

      \put(15, 7.5){\vector(1, 1){5}}
      \put(15, 7.5){\vector(1, -1){5}}
      \put(35, 2.5){\vector(1, 1){5}}
      \put(35, 12.5){\vector(1, -1){5}}
    \end{picture}
    \label{fig:threenodes}
  }

  \subfigure[Several front-ends]{
    \begin{picture}(55, 20)
      \put(0, 15){\dashbox(15, 5){Browser}}
      \put(20, 0){\framebox(15, 5){Front-end}}
      \put(20, 15){\framebox(15, 5){Front-end}}
      \put(40, 15){\framebox(15, 5){Back-end}}

      \put(22.5, 12.5){$\circ$}
      \put(22.5, 10){$\circ$}
      \put(22.5, 7.5){$\circ$}

      \put(15, 17.5){\vector(1, 0){5}}
      \put(15, 17.5){\vector(1, -3){5}}
      \put(35, 17.5){\vector(1, 0){5}}
      \put(35, 2.5){\vector(1, 3){5}}
    \end{picture}
    \label{fig:frontends}
  }

  \subfigure[Several front-ends and back-ends (slaves)]{
    \begin{picture}(75, 20)
      \put(0, 15){\dashbox(15, 5){Browser}}
      \put(20, 0){\framebox(15, 5){Front-end}}
      \put(20, 15){\framebox(15, 5){Front-end}}
      \put(40, 10){\framebox(15, 10){\minibox{Back-end \\ Master}}}
      \put(60, 0){\framebox(15, 5){Slave}}
      \put(60, 15){\framebox(15, 5){Slave}}

      \put(22.5, 12.5){$\circ$}
      \put(22.5, 10){$\circ$}
      \put(22.5, 7.5){$\circ$}
      \put(62.5, 12.5){$\circ$}
      \put(62.5, 10){$\circ$}
      \put(62.5, 7.5){$\circ$}

      \put(15, 17.5){\vector(1, 0){5}}
      \put(15, 17.5){\vector(1, -3){5}}
      \put(35, 17.5){\vector(1, 0){5}}
      \put(35, 2.5){\vector(1, 3){5}}
      \put(55, 17.5){\vector(1, 0){5}}
      \put(55, 17.5){\vector(1, -3){5}}
    \end{picture}
    \label{fig:frontendbackends}
  }

  \subfigure[Legend]{
    \begin{picture}(100, 20)
      \put(0, 5){\dashbox(15, 5){}}
      \put(0, 0){Non-system interaction}

      \put(40, 5){\framebox(15, 5)}
      \put(40, 0){Node}

      \put(65, 5){\vector(1, 0){10}}
      \put(65, 0){Connection flow}

      \put(95, 5){$\circ$}
      \put(95, 0){n-times}
    \end{picture}
  }

  \caption{Different architectural ways to provision nodes.}
  \label{fig:BankManager}
\end{figure}


Some examples of provisioning topologies are illustrated in \citefig{BankManager}.
Each example includes a \texttt{browser} to visualize application flow,
\texttt{front-end} visualizes executable logic and \texttt{back-end} represents database.
It is possible to have both \texttt{front-end} and \texttt{back-end} on the same node, 
as shown in \citefig{singlenode}.
When the topology have several \texttt{front-ends} a \texttt{load balancer} is used
to direct traffic between \texttt{browser} and \texttt{front-end}.
The \texttt{load balancer} could be a node like the rest, but in this cloud-based scenario
it is actually a cloud service, which is also why it is graphically different.
In \citefig{twonodes} \texttt{front-end} is separated from \texttt{back-end},
this introduces the flexibility of increasing computation power on the \texttt{front-end} node while spawning more
storage on the \texttt{back-end}.
For applications performing heavy computations it can be beneficial to distribute the workload between several
\texttt{front-end} nodes as seen in \citefig{threenodes}, the number of \texttt{front-ends} can be linearly increased
$n$ number of times as shown in \citefig{frontends}.
\emph{BankManager} is not designed to handle several \texttt{back-ends} because of \myac{RDBMS},
this can be solved at the database level with master and slaves (\citefig{frontendbackends}).

\paragraph{Execution.}

The main goal of the scenario was to successfully deploy \emph{BankManager}
on a given set of providers with a given set of topologies.
And to achieve such deployment it was crucial to perform cloud provisioning.
The providers chosen were 
\begin{ii}
  \iitem \myac{AWS}~\cite{aws} and
  \iitem Rackspace~\cite{rackspace}.
\end{ii}
These are strong providers with a respectable amount of customers, as two of the leaders in cloud computing
[\todo{source}].
They also have different graphical interfaces, APIs and toolchains which makes them suitable
for a scenario researching multicloud challenges.

The topology chosen for this scenario was that of three nodes\citefig{threenodes}.
This topology is advance enough that it needs a \texttt{load balancer} in front of two
\texttt{front-end} nodes, and yet the simplest topology of the ones that benefits from a \texttt{load balancer}.
It is important to include most of the technologies and services that needs testing.

To perform the actual provisioning a set of primitive Bash-scripts were developed.
These scripts were designed to automate a full deployment on a two-step basis.
First step was to provision instances:
\begin{itemize}
  \item Authenticate against provider.
  \item Create instances.
  \item Manually write down IP addresses of created instances.
\end{itemize}
The second step was deployment:
\begin{itemize}
  \item Configure \emph{BankManager} to use one of provisioned instances IP address for database.
  \item Build \emph{BankManager} into a \myac{WAR}-file.
  \item Authenticate to instance using \myac{SSH}.
  \item Remotely execute commands to install required third party software such as Java and PostgreSQL.
  \item Remotely configure third party software.
  \item Inject \myac{WAR}-file into instances using \myac{SFTP}.
  \item Remotely start \emph{BankManager}.
\end{itemize}
The scripts were provider-specific so one set of scripts had to be made for each provider.
Rackspace had at that moment no command-line tools, so a \myac{REST} client had to be constructed.

\section{Challenges}

From this research it became clear that there were multiple challenges to address
when deploying applications to cloud infrastructure.
This thesis is scoped to cloud provisioning, but the goal of this provisioning is to 
enable a successful deployment. 
It was therefore crucial to involve a full deployment in the scenario to discover
important challenges.

\paragraph{Complexity.} 

The first challenge encountered was to simply 
authenticate and communicate with the cloud. 
The two providers had different approaches, \myac{AWS}~\cite{aws} 
had command-line tools built from their Java APIs,
while Rackspace~\cite{rackspace} had no tools beside the API language bindings.
So for \myac{AWS} the Bash-scripts could do callouts to the command-line interface 
while for Rackspace the public \myac{REST} API had to be utilized.
This emphasized the inconsistencies between providers, 
and resulted in an additional tool being introduced to handle requests.

As this emphasizes the complexity even further it also stresses engineering 
capabilities of individuals.
It would be difficult for non-technical participants to fully understand and give comments
or feedback on the topology chosen since important information got hidden behind
complex commands.

\paragraph{Feedback on failure.}
Debugging the scripts were also a challenging task, since they fit together by
sequential calls and printed information based on Linux and Bash commands such as 
\emph{grep} and \emph{echo}.
Error messages from both command-line and \myac{REST} interfaces were essentially muted away.
If one specific script should fail it was difficult to know 
\begin{ii}
  \iitem which script failed, 
  \iitem at what step it was failing and 
  \iitem what was the cause of failure
\end{ii}.

\paragraph{Multicloud.}

Once able to provision the correct amount of nodes with desired properties
on the first provider it became clear that mirroring the setup to the other provider 
was not as convenient as anticipated.
There were certain aspects of vendor lock-in, so each script was hand-crafted for specific providers.
The most noticeable differences would be
\begin{ii}
  \iitem different ways of defining instance sizes,
  \iitem different versions, distributions or types of operating systems (\emph{images}),
  \iitem different way of connection to provisioned instances
\end{ii}.
The lock-in situations can in many cases have financial implications where for example
a finished application is locked to one provider and this provider increases tenant costs.
Or availability decreases and results in decrease of service uptime damaging revenue.

\paragraph{Reproducibility.}

The scripts provisioned nodes based on command-line arguments
and did not persist the designed topology in any way, 
this made topologies cumbersome to reproduce.
If the topology could be persisted in any way, for example serialized files,
it would be possible to reuse these files at a later time.
The persisted topologies could also be reused on other clouds making a 
similar setup at another cloud provider, or even distribute the setup
between providers.

\paragraph{Shareable.}

Since the scripts did not remember a given setup it was impossible 
to share topologies ``as is'' between coworkers.
It is important that topologies can be shared because direct input from individuals
with different areas of competence can increase quality.
If the topology could be serialized into files these files could also be interpreted
and loaded into different tools to help visualizing and editing.

\paragraph{Robustness.}

There were several ways the scripts could fail and most errors were ignored.
They were made to search for specific lines in strings returned by the APIs,
if these strings were non-existent the scripts would just continue regardless
of complete dependency to information within the strings.
A preferable solution to this could be transactional behavior with rollback functionality
in case an error should occur, or simply stop the propagation
and throw exceptions that can be handled on a higher level.

\paragraph{Metadata dependency.}

The scripts were developed to fulfill a complete deployment,
including 
\begin{ii}
  \iitem provisioning instances, 
  \iitem install third party software on instances,
  \iitem configure instances and software,
  \iitem configure and upload \myac{WAR}-file and
  \iitem deploy and start the application from the \myac{WAR}-file
\end{ii}.
In this thesis the focus is aimed at provisioning, but it proved important to temporally 
save run-time specific metadata to successfully deploy the application.
In the \emph{BankManager} example the crucial metadata was information needed to connect 
front-end nodes with the back-end node, but other deployments is likely to need the same 
or different metadata for other tasks.
This metadata is collected in \iii{1}, and used in \iii{3} and \iii{4}.

\chapter{Requirements to solution (with table)}

\todo{
  \begin{itemize}
    \item Copy in my existing table from the essay \\
      \question{Requirements = challenge = problem?}
  \end{itemize}
}


\part{Contribution}

\chapter{Vision, concepts and principles}

\mychapter{design}{Analysis and design - CloudML}
\begin{figure}
  \tikzstyle{class}=[rectangle, draw=black, rounded corners, fill=white!40, drop shadow,
  text centered, anchor=north, text=black, text width=3.5cm, top color=white, bottom color=black!10]
  \tikzstyle{m@rt}=[rectangle, draw=black, rounded corners, fill=gray, drop shadow,
  text centered, anchor=north, text=white, text width=3.5cm]
  \tikzstyle{arrow}=[->, >=open triangle 90]
  \tikzstyle{line}=[-]
  \tikzstyle{aggregate}=[>-, >=diamond]

  \begin{center}
      \begin{tikzpicture}[scale=0.6, transform shape]

        \node (UserLibrary) [class, rectangle split, rectangle] {
            \textbf{UserLibrary}
          };
        \node (CloudMLEngine) [class, rectangle, rectangle split parts=3, below=of UserLibrary] {
                \textbf{CloudMLEngine}
                \nodepart{second}
                \nodepart{third}+build
          };
        \node (AuxNode01) [text width=4cm, below=of CloudMLEngine] {};

        \node (Account) [class, rectangle split, rectangle split parts=2, left=of CloudMLEngine, label=left:*] {
                \textbf{Account}
                \nodepart{second}+name: String
            };
        \node (Credential) [class, rectangle, rectangle, below=of Account, label=67.5:1] {
                \textbf{Credential}
            };
        \node (Password) [class, rectangle split, rectangle split parts=2, below=of Credential, xshift=-4cm] {
                \textbf{Password}
                \nodepart{second}+identity: String \\ ++credential: String
            };
        \node (KeyPair) [class, rectangle split, rectangle split parts=2, below=of Credential] {
                \textbf{KeyPair}
                \nodepart{second}+public: String
            };

        \node (Connector) [class, rectangle, rectangle, right=of CloudMLEngine, label=175:*, label=67.5:1] {
                \textbf{Connector}
            };
        \node (AmazonEC2) [class, rectangle, rectangle, below=of Connector, xshift=-2cm] {
                \textbf{AmazonEC2}
            };
        \node (Rackspace) [class, rectangle, rectangle, below=of Connector, xshift=2cm] {
                \textbf{Rackspace}
            };

        \node (System) [m@rt, rectangle, rectangle, below=of AuxNode01] {
                \textbf{System}
            };
        \node (RuntimeInstance) [m@rt, rectangle, rectangle, below=of System, yshift=-1cm, label=67.5:*] {
                \textbf{RuntimeInstance}
            };
        \node (RuntimeProp) [m@rt, rectangle, rectangle, left=of RuntimeInstance, label=5:*] {
                \textbf{RuntimeProp}
            };
        \node (PublicIP) [m@rt, rectangle split, rectangle split parts=2, below=of RuntimeProp, xshift=-2cm] {
                \textbf{PublicIp}
                \nodepart{second}+Value: Address
            };
        \node (PrivateIP) [m@rt, rectangle split, rectangle split parts=2, below=of RuntimeProp, xshift=2cm] {
                \textbf{PrivateIp}
                \nodepart{second}+Value: Address
            };

        \node (Template) [class, rectangle split, rectangle split parts=2, right=of System, label=-5:*] {
                \textbf{Template}
                \nodepart{second}+name: String
            };
        \node (Node) [class, rectangle split, rectangle split parts=2, below=of Template, label=67.5:*] {
                \textbf{Node}
                \nodepart{second}+id: String
            };
        \node (Property) [class, rectangle, rectangle, below=of Node, label=67.5:*] {
                \textbf{Property}
            };
        \node (Location) [class, rectangle split, rectangle split parts=2, below=of Property, yshift=-0.25cm] {
                \textbf{Location}
                \nodepart{second}+value: String
            };
        \node (Disk) [class, rectangle split, rectangle split parts=2, left=of Location] {
                \textbf{Disk}
                \nodepart{second}+min: String
            };
        \node (Core) [class, rectangle split, rectangle split parts=2, left=of Disk] {
                \textbf{Core}
                \nodepart{second}+min: String
            };
        \node (RAM) [class, rectangle split, rectangle split parts=2, left=of Core] {
                \textbf{RAM}
                \nodepart{second}+min: String
            };

        \draw[arrow] (Password) -- ++(0, 1.25) -| (Credential);
        \draw[arrow] (KeyPair) -- ++(0, 1.25) -| (Credential);

        \draw[arrow] (AmazonEC2) -- ++(0, 0.75) -| (Connector);
        \draw[arrow] (Rackspace) -- ++(0, 0.75) -| (Connector);

        \draw[arrow] (RAM) -- ++(0, 0.9) -| (Property);
        \draw[arrow] (Core) -- ++(0, 0.9) -| (Property);
        \draw[arrow] (Disk) -- ++(0, 0.9) -| (Property);
        \draw[arrow] (Location) -- ++(0, 0.9) -| (Property);

        \draw[arrow] (PublicIP) -- ++(0, 0.9) -| (RuntimeProp);
        \draw[arrow] (PrivateIP) -- ++(0, 0.9) -| (RuntimeProp);

        \draw[line] (Account) -- ++(0, 1) -| (Connector);

        \draw[aggregate] (UserLibrary) -| (9, -1) -- (9, -5) -| (Template.east);
        \draw[aggregate] (UserLibrary) -| (Account.west);

        \draw[aggregate] (Account) -- ++(0, -1) -| (Credential.north);

        \draw[aggregate] (CloudMLEngine.east) -| (Connector.west);

        \draw[aggregate] (Template) -- ++(0, -1) -| (Node.north);
        \draw[aggregate] (Node) -- ++(0, -1) -| (Property.north);

        \draw[line] (Node.west) -- ++(0, 1) -| (System.east);
        \draw[line] (Template.west) -| (System.east);

        \draw[aggregate] (System) -- ++(0, -1) -| (RuntimeInstance.north);

        \draw[aggregate] (RuntimeInstance) -| (RuntimeProp.east);

    \end{tikzpicture}
  \end{center}
  \caption{Architecture of CloudML}
  \label{fig:architecture}
\end{figure}

\begin{figure}
  \begin{center}
    \begin{tikzpicture}[scale=0.7, transform shape]
      \node (Instance) [class] { \textbf{Instance} };
    \end{tikzpicture}
  \end{center}
  \caption{Scenario1}
  \label{fig:scenario1}
\end{figure}


\begin{figure}[tb]
  \begin{center}
    \subfigure[Template with nodes.] {
      \begin{tikzpicture}[scale=0.7, transform shape]
        \node (Template) [class, rectangle split, rectangle split parts=2] {
          \textbf{:Template}
          \nodepart{second}name=``template1''
        };
        \node (Node02) [class, rectangle split, rectangle split parts=2, right=of Template] {
          \textbf{:Node}
          \nodepart{second}name=``node2'' \\ cores=2
        };
        \node (Node01) [class, rectangle split, rectangle split parts=2, above=of Node02] {
          \textbf{:Node}
          \nodepart{second}name=``node1'' \\ cores=2
        };
        \node (Node03) [class, rectangle split, rectangle split parts=2, below=of Node02] {
          \textbf{test1:Node}
          \nodepart{second}name=``node3'' \\ disk=2000
        };

        \draw[line] (Template) -- (Node01);
        \draw[line] (Template) -- (Node02);
        \draw[line] (Template) -- (Node03);
      \end{tikzpicture}
      \label{fig:scenario2-1}
    }

    \subfigure[Instance.] {
      \begin{tikzpicture}[scale=0.7, transform shape]
        \node (Instance01) [class, text width=5cm, rectangle split, rectangle split parts=2] {
          \textbf{:Instance}
          \nodepart{second}name=``node1'' \\ cores=2 \\ templateName=``template1''
        };
        \node (Instance02) [class, text width=5cm, rectangle split, rectangle split parts=2, right=of Instance01] {
          \textbf{:Instance}
          \nodepart{second}name=``node2'' \\ cores=2 \\ templateName=``template1''
        };
        \node (Instance03) [class, text width=5cm, rectangle split, rectangle split parts=2, right=of Instance02] {
          \textbf{:Instance}
          \nodepart{second}name=``node4'' \\ disk=2000 \\ templateName=``template1''
        };
      \end{tikzpicture}
      \label{fig:scenario2-2}
    }
  \end{center}
  \caption{Scenario reimplemented with three nodes.}
  \label{fig:scenario2}
\end{figure}


\begin{figure}[tb]
  \begin{sequencediagram}
    \newthread{u}{:User}
    \newthreadhack{c}{:CloudML}
    \newinst{r}{:RuntimeInstance}
    \newthread{a}{:AWS}
    
    \begin{call}{u}{build(\texttt{account},List(\texttt{template}))}{c}{List(\texttt{RuntimeInstance})}
      \begin{call}{c}{Initialize()}{r}{}
      \end{call}
    \end{call}
    \begin{messcall}{c}{provision node}{a}
    \end{messcall}
    
    \begin{call}{u}{getStatus()}{r}{\emph{"Building"}}
    \end{call}
    
    \begin{messcall}{a}{status(\emph{"Starting"})}{c}
    \end{messcall}
    
    \begin{messcall}{c}{update(\emph{"Starting"})}{r}
    \end{messcall}
    
    
    \begin{call}{u}{getStatus()}{r}{\emph{"Starting"}}
    \end{call}
  \end{sequencediagram}
  
  \caption{CloudML asynchronous provisionning process (Sequence diagram).}
  \label{fig:sequencediagram}
\end{figure}


\todo{Read over, copied from old vision} \\
In this chapter the core approach and steps on the research to implementing CloudML will be described.
In the previous chapters, \citechap{challenges} and \citechap{requirements}, 
challenges and requirements were identified and described.
In this part of the thesis (\emph{contribution}) the challenges will be addressed 
by applying and implementing the requirements.

There are four main steps from start of the problem to an functional implementation.
\begin{enumerate}
  \item Identify and research in \emph{state of the art}.
  \item Recognize \emph{challenges}.
  \item Determine \emph{requirements} based on \emph{challenges}.
  \item \emph{Analyze} solutions, tools and procedure to implement \emph{requirements}.
  \item \emph{Implement} a solution based on \emph{analyzed} results.
\end{enumerate}
Of these steps step $1$ (one) to $3$ (three) are already covered by \citechap{state-of-the-art}
, \citechap{challenges} and \citechap{requirements}.
This chapter is an intermediate chapter which will introduce the chapter of \citechap{design}
and \citechap{implementation}.

\todo{read over}

For the sake of tidiness, clarity and technical limitations it 
should not be possible to define cross-provider (\emph{multicloud}) nodes in a single topology.
This itself does not mean CloudML will not support multicloud provisioning,
instead such functionality is achieved by utilizing more than one template,
which will not retain a full multicloud deployment.

\begin{center}
\line(1,0){250}
\end{center}

In the previous chapter, \citechap{vision}, procedure of requirements analyzing
were described.
In this chapter these procedures will be described including their results.
The meta model for CloudML will also be presented, through a specific scenario.

\section{Analysis of requirements}

Here the analysis procedure for each of the requirements 
that needed analysis according to \citechap{vision} will be described.

\subparagraph{Dependencies.}

As stated in \citechap{vision} the dependency between four of the requirements
are strong, this is considered for each of the following requirements.
Since \texttt{underlying technology} is a central dependency (\citefig{requirement-dependencies})
this requirements is weighted more than the others.

\paragraph{Underlying technology.}

For \emph{programming language} and \emph{application environment} the following were considered.
\begin{ii}
  \iitem JavaScript (Node.js) [\todo{Need source? Got 2 articles}],
  \iitem Java (JDK),
  \iitem Scala (JDK),
  \todo{
    \iitem Python and
    \iitem C\# (.NET).
  }
\end{ii}
Preceding \iii{2} to \iii{5} are well-known languages and environments recognized by both the enterprise world as well as the academic world.
Preceding \iii{1} is a newer and more unknown type of technology which has yet to 
make a large footprint in the two worlds. 
It is based on the \myac{JIT}-powered JavaScript engine V8 created by Google and follow a 
CommonJS-based module pattern over an event-driven architecture.
Preceding \iii{1} were brought in as an consideration because of the abilities to operate
with \myac{JSON}, its aim at web development on the cloud and the modernity.

\paragraph{Model-driven approach.}

\paragraph{Lexical templates.}

\paragraph{Models@run.time.}

\section{Meta model.}

The meta model for CloudML is visualized in~\citefig{architecture}. 
\texttt{CloudMLEngine} is the main entry point, it has the method \texttt{build}
which is used to initialize provisioning.
\texttt{Property} have four children but is designed to be extendable in case
new types of properties should be included. The same design principle
is applied to \texttt{RuntimeProp}.
\texttt{UserLibrary} visualizes that \texttt{Account} and \texttt{Template} are 
physical parts maintainable by the user.

\paragraph{Scenario introduction.}
CloudML is introduced by using two scenarios where ``Alice'' is provisioning the 
\emph{BankManager} from \citechap{challenges} to AWS \emph{Elastic Compute Cloud}~(EC2)
using the topology shown in~\citefig{singlenode} and~\citefig{threenodes}.
It is compulsory that she possesses an AWS account in advance of the scenario.
She will retrieve security credentials for account 
and associate them with \texttt{Password} in \citefig{architecture}.
\texttt{Credential} is used to authenticate her to supported providers through \texttt{Connector}.
The characteristics Alice choose for her \texttt{Nodes} and \texttt{Properties} are fitted
for the chosen topology.
All \texttt{Properties} are optional and thus Alice does not have to define them all.

\paragraph{Scenario with one single node.}
The first scenario Alice want to establish is a single node based one~(\citefig{singlenode}).
Since this single node will handle both computation and storage Alice decides to 
increase capabilities of both processing (number of \texttt{Cores}) and 
\texttt{Disk} size on the \texttt{Node}.

\paragraph{Scenario with three nodes.}
The second scenario is based on~\citefig{threenodes} with two more nodes than in the first scenario.
Alice models the appropriate \texttt{Template} consisting
of three \texttt{Nodes}.
by increasing amount of \texttt{Cores}, and increased \texttt{Disk} for back-end \texttt{Node}.

\paragraph{Provisioning.}
With these models Alice can initialize provisioning by calling 
\texttt{build} on \texttt{CloudMLEngine},
and this will start the asynchronous job of configuring and creating \texttt{Nodes}.
When connecting front-end instances of \emph{BankManager} to back-end instances Alice must 
be aware of the back-ends \texttt{PrivateIP} address, which she will retrieve from CloudML
during provisioning according to \emph{models@run.time}~(M@RT) approach.
\texttt{RuntimeInstance} is specifically designed to complement \texttt{Node} with \texttt{RuntimeProperties},
as \texttt{Properties} from \texttt{Node} still contain valid data.
When all \texttt{Nodes} are provisioned successfully and sufficient metadata are gathered
Alice can start the deployment, CloudML has then completed its scoped task of provisioning.
Alice could later decide to use another provider, either as replacement or complement to her current setup,
because of availability, financial benefits or support.
To do this she must change the provider name in \texttt{Account} and call \texttt{build} on \texttt{CloudMLEngine} again,
this will result in an identical topological setup on a supported provider.

\paragraph{And we saw?}

\section{Actors model}
Provisioning nodes is by its nature an asynchronous action that can take minutes to execute,
therefore CloudML relied on the actors model~\cite{actors:haller07}.
With this asynchronous solution CloudML got concurrent communication with nodes under provisioning.
The model is extended by adding a callback-based pattern allowing each node to provide 
information on property and status changes.
Developers exploring the implementation can then choose to ``listen'' for updating events from each node,
and do other jobs / idle while the nodes are provisioned with the actors model.
The terms are divided for a node before and under provisioning, the essential is to introduce 
\emph{M@RT} to achieve a logical separation.
When a node is being propagated it changes type to \texttt{RuntimeInstance}, 
which can have a different \emph{states} such as \emph{Configuring}, \emph{Building}, \emph{Starting} and \emph{Started}.
When a \texttt{RuntimeInstance} reaches \emph{Starting} state the provider has guaranteed its existence, including
the most necessary metadata, when all nodes reaches this state the task of provisioning is concluded.

\mychapter{implementation}{Implementation/realization - cloudml-engine}
\begin{figure}[tb]
  \begin{center}
    \subfigure[Application modules]{
      \begin{tikzpicture}[scale=0.8, transform shape]
        \node (AuxNode01) [text width=4cm] {};
        \node (Engine) [class, left=of AuxNode01, rectangle split, rectangle split parts=2] { 
          \textbf{Engine} 
          \nodepart{second}Entry point. Orchestration.
        };
        \node (AuxNode02) [left=of Engine] {};
        \node (Kernel) [class, below=of AuxNode01, rectangle split, rectangle split parts=2] { 
          \textbf{Kernel} 
          \nodepart{second}Node domains. Converts JSON to Node Entities.
        };
        \node (Repository) [class, above=of AuxNode01, rectangle split, rectangle split parts=2] { 
          \textbf{Repository} 
          \nodepart{second}Instance domains. Convert Nodes to Instances.
        };
        \node (Cloud-Connector) [class, right=of AuxNode01, rectangle split, rectangle split parts=2] { 
          \textbf{Cloud-Connector} 
          \nodepart{second}Connects to providers (jclouds).
        };

        \draw [arrow] (Engine) -- (Kernel);
        \draw [arrow] (Engine) -- (Cloud-Connector);
        \draw [arrow] (Engine) -- (Repository);
        \draw [arrow] (Cloud-Connector) -- (Kernel.east);
        \draw [arrow] (Cloud-Connector) -- (Repository.east);
        \draw [arrow] (Repository) -- (Kernel);
        \draw [extend] (AuxNode02) -- (Engine);
      \end{tikzpicture}
    }

    \subfigure[Legend]{
      \begin{tikzpicture}[scale=0.7, transform shape]
        \node (Module) [class, label=below:Application modules] { Module };

        \node (AuxNode01) [right=of Module] {};
        \node (AuxNode02) [right=of AuxNode01] {};
        \node (AuxNode03) [right=of AuxNode02] {};
        \node (AuxNode04) [right=of AuxNode03] {};

        \draw[arrow] (AuxNode01) -- node[below] {Dependency} (AuxNode02);
        \draw[extend] (AuxNode03) -- node[below] {Entry} (AuxNode04);
      \end{tikzpicture}
    }
  \end{center}
  \caption{Architecture of cloudml-engine}
  \label{fig:cloudml-engine}
\end{figure}

\begin{figure}

  \begin{center}
    \begin{tikzpicture}
      \stickman{User}{User}

      \node[box, right=of User, minimum width=3.8cm, minimum height=3.5cm, xshift=1cm, yshift=-1cm] (System) {};
      \node[box, right=of User, xshift=2cm, yshift=-0cm] (Engine) {Engine};
      \node[box, below=of Engine, yshift=0.5cm] (Cloud-Connector) {Cloud-Connector};
      \node[box, below=of Cloud-Connector, yshift=0.5cm] (RuntimeInstance) {RuntimeInstance};

      \node (Rackspace) [tcloud, right=of System, xshift=0.5cm, yshift=1.5cm] {Rackspace};
      \node (EC2) [tcloud, below=of Rackspace] {EC2};

      \draw[arrow] (User.east) -- (Engine.west);
      \draw[arrow] (Engine.south) -| (Cloud-Connector.north);

      \draw[arrow] (Cloud-Connector.east) -- (Rackspace.west);
      \draw[arrow] (Cloud-Connector.east) -- (EC2.west);

      \draw[arrow] (Cloud-Connector.south) -| (RuntimeInstance.north);
      \draw[arrow] (RuntimeInstance.west) -- (User.east);

      \node (Label02) [below=of System, xshift=1.2cm] {Cloudml-engine};
      \node [right=of Label02, xshift=1.2cm] {Cloud providers};

    \end{tikzpicture}
  \end{center}
  \caption{Usage flow in cloudml-engine}
  \label{fig:cloudml-engine-flow}
\end{figure}

\begin{figure}
  \begin{center}
    \begin{minted}[mathescape,
                   linenos,
                   numbersep=5pt,
                   gobble=2,
                   frame=lines,
                   framesep=2mm]{xml}

  <repositories>
   <repository>
    <id>cloudml-engine</id>
    <url>
     https://repository-eirikb.forge.cloudbees.com/release
    </url>
   </repository>
  </repositories>
  <dependencies>
   <dependency>
    <groupId>no.sintef</groupId>
    <artifactId>engine</artifactId>
    <version>0.1</version>
   </dependency>
  </dependencies>
    \end{minted}
  \end{center}
  \caption{Example Maven configureation section to include cloudml-engine}
  \label{fig:pom-example}
\end{figure}


\begin{figure}[tb]
  \begin{center}
    \begin{minted}[mathescape,
                   linenos,
                   numbersep=5pt,
                   frame=lines,
                   framesep=2mm]{scala}

import no.sintef.cloudml.engine.Engine
...
val runtimeInstances = Engine(account, List(template))
    \end{minted}
  \end{center}
  \caption{Example client (Scala) callout to cloudml-engine.}
  \label{fig:cloudml-engine-usage}
\end{figure}



The envision and design of CloudML is implemented as a proof-of-concept project \emph{cloudml-engine}.
The project is split into four different modules~(\citefig{cloudml-engine}). 
Each module serves a logical task of CloudML.
This chapter will go into depths of technologies and structures of the implementation.

\section{Technologies}

\emph{Cloudml-engine} is based on state-of-the-art technologies that appeal to the academic community.
Technologies chosen for \emph{cloudml-engine} are not of great importance to the concept of CloudML itself,
but it still important to understand which technologies were chosen, what close alternatives exists
and why they were chosen.

\paragraph{Language.} 
\emph{Cloudml-engine} is written in Scala, a multi-paradigm JVM based programming language.
This language was chosen because JVM is a popular platform, and then especially Java.
Scala is compatible with Java and Java can interact with libraries written in Scala as well.
The reason not to use plain Java was because Scala is an appealing state-of-the-art language that emphasizes 
on functional programming which is leveraged in the implementation.
Scala also has a built in system for Actors model~\cite{actors:haller07} which is utilized in the implementation.

\paragraph{Lexical format.}
For the lexical representation of CloudML \emph{JavaScript Object Notation}~(JSON) was chosen.
JSON is a web-service friendly, human-readable data interchange format and an alternative to XML.
This format was chosen because of popularity in the cloud community \todo{source}
and its usage area as data transmit format between servers and web applications.
This means \emph{cloudml-engine} can be extended to work as a RESTFul web-service server.

The JSON format is parsed in Scala using the lift-json parser which provides implicit
mapping to Scala case-classes. This library is part of the lift framework,
but can be included as an external component without additional lift-specific dependencies.
GSON was considered as an alternative, but mapping to Scala case-classes was not as 
fluent compared to lift-json.

\paragraph{Automatic build system.}
There are two main methods used to build Scala programs, either using a Scala-specific tool called 
\emph{Scala Build Tool}~(SBT) or a more general tool called Maven. 
For \emph{cloudml-engine} to have an academic appeal it were essential to choose the technology
with most closeness to Java, hence Maven was chosen.
Maven support modules which were used to split \emph{cloudml-engine} into the appropriate 
modules as shown in~\citefig{cloudml-engine}. 
The dependency system in Maven between modules is used to match the dependencies outlined in~\citefig{cloudml-engine}.
Parts of a dependency reference in a Maven configuration can be seen in~\citefig{pom-example}.

\paragraph{Cloud connection.}
The bridge between \emph{cloudml-engine} and cloud providers is an important aspect of the application, and as a requirement
it was important to use an existing library to achieve this connection.
Some libraries have already been mentioned in the \emph{APIs} section in~\citechap{state-of-the-art},
of these only \emph{jclouds} is based on Java-technologies and therefore suites \emph{cloudml-engine}.
Jclouds uses Maven for building as well, and is part of Maven central which makes 
it possible to add jclouds directly as a module dependency.
Jclouds contains a template system which is used through code directly, this is utilized 
to map CloudML templates to jclouds templates.

\paragraph{Distribution.}
\emph{Cloudml-engine} is not just a proof-of-concept for the sake of conceptual assurance, but it is 
also a running, functional library which can be used by anyone for testing or considerations.
Beside the source repository\cite{cloudml-engine} the library is deployed to a remote repository
\cite{cloudbees-cloudml-engine} as a Maven module.
This repository is provided by CloudBees, 
how to include the library is viewable in~\citefig{pom-example}.

\paragraph{Actors.}
As mentioned earlier \emph{cloudml-engine} utilizes the actors model through Scala,
this approach is used to achieve asynchronous provisioning.
This is important as provisioning can consume up to minutes for each instance.
Beside the standard model provided by Scala \emph{cloudml-engine} uses
a callback-based pattern to inform users of the library when instance statues
are updated and properties are added.

\section{Modules and application flow}

\emph{Cloudml-engine} is divided into four main modules~\citefig{cloudml-engine}.
This is to distribute workload and divide \emph{cloudml-engine} into logical parts for each task.

\paragraph{Engine.} The main entry point to the application, this is a Scala Object used to initialize
provisioning.
Interaction between \texttt{user} and \texttt{Engine} is visible in \citefig{cloudml-engine-flow} 
where the user will initialize provisioning by calling \texttt{Engine}.
\texttt{Engine} will also do orchestration between the three other modules
as shown in \citefig{cloudml-engine}.
Since \texttt{Cloud-Connector} is managed by \texttt{Engine} other actions against 
instances are done through \texttt{Engine}.
The first versions of \emph{cloudml-engine} did not use \texttt{Engine} as orchestrator but
instead relied on each module to be a sequential step, this proved to be harder to maintain
and also introduced cyclic dependencies.

\paragraph{Kernel.} \texttt{Kernel} contains CloudML specific entities such as Node and Template.
The logical task of \texttt{Kernel} is to map JSON formatted strings to \texttt{Templates} including \texttt{Nodes}.
This is some of the core parts of the DSL, hence it is called \emph{\texttt{Kernel}}.
\texttt{Accounts} are separate parts that are parsed equally as \texttt{Templates},
 but by another method call. All this is transparent for users as all data will
be provided directly to \texttt{Engine} which will handle the task
of calling \texttt{Kernel} correctly.

\paragraph{Repository.} Has \texttt{Instance} entities, these are equivalent to \texttt{Nodes} in \texttt{Kernel},
but are specific for provisioning. Repository will do a mapping from \emph{Nodes} (including \emph{Template})
to \emph{Instances}. Future versions of \texttt{Repository} will also do some logical superficial validation
against \emph{Node} properties, for instance at the writing moment it is not possible to 
demand LoadBalancers on Rackspace for specific geographical locations.

\paragraph{Cloud-Connector.} is the module bridging between \emph{cloudml-engine} and providers.
It does not contain any entities, and only does logical code. 
It is built to support several libraries and interface these. At the moment it only implements the earlier
mentioned library jclouds.

\chapter{Validation on example/experiments - BankManager}

\todo{
  \begin{itemize}
    \item How BankManager proves concepts of the templates (subsection 1) with cloudml-engine
  \end{itemize}
}


\part{Conclusion}

\mychapter{conclusions}{Conclusions}

In this thesis four main parts have been presented.
First the background part introducing the domain of cloud computing and model-driven engineering.
Then the second part highlighting sets of technologies, frameworks, ideas and 
\myac{API}s which are currently used in the two domains.
Third the challenges with these solutions are stressed, as well as a set of \emph{requirements}
CloudML must fulfill to tackle these challenges.
Lastly, CloudML is presented, in three phases,
\begin{ii}
  \iitem vision,
  \iitem design and
  \iitem implementation.
\end{ii}

In the vision chapter the core idea of CloudML were introduced,
and even means to tackle \citereq{m@rt} were outlined through pure vision.
The design chapter stated how CloudML should be built up,
what the meta-model should look like,
what underlying technologies should be used.
All through a scenario where Alice performs provisioning.
In this chapter the means to tackle \citereq{m@rt} are reinforced through
the view of design.
The requirements of \citereq{foundation} and \citereq{software-reuse}
are addressed through what underlying technology to use and alternatives.
Lastly the implementation chapter outline how CloudML is implemented
as \emph{cloudml-engine}, and how this solution is built up.
Both \citereq{mda} and \citereq{lexical-template} are tackled in this chapter
by concretely choosing data format and syntax based on the design chapter.

\section{Results}
\begin{table}
  \begin{tabular*}{\textwidth}{@{\extracolsep{\fill}}| l | l | l | l | l | l |}
      \hline
        \textbf{State of the art} & 
        \textbf{\citereq{software-reuse}} & 
        \textbf{\citereq{foundation}} & 
        \textbf{\citereq{mda}} & 
        \textbf{\citereq{m@rt}} & 
        \textbf{\citereq{lexical-template}} \\
      \hline
     Amazon CloudFormation & & & Yes & No & Yes \\ \hline
     CA Applogic & & & Yes & & No \\ \hline
     jclouds & & Yes & Partly & No & No \\ \hline
     mOSAIC & Yes & Yes & No & No & No \\ \hline
     Amazon Beanstalk & Yes & Yes & No & No & No \\ \hline
     CloudML & Yes & Yes & Yes & Yes & Yes \\ \hline
  \end{tabular*}
  \caption{Comparing selected elements from \citechap{state-of-the-art} with requirements.}
  \label{table:requirements-comparison}
\end{table}



The implementation is validated through an experiment where
it is physically executed against two providers, \myac{AWS} and Rackspace.
This experiment concludes the work put into CloudML at this point to be successful.

The requirements from \citechap{requirements} are compared against selected
technologies and frameworks from \citechap{state-of-the-art}.
These comparisons are expressed in \citetable{requirements-comparison}.

These are the ``\emph{key results}'' composed by this thesis:
\begin{description}
  \item[meta-model.]
    A meta-model was designed, with model-driven engineering in mind.
  \item[Engine.]
    An engine capable of provisioning nodes on a set of supported cloud providers
    was implemented.
  \item[Lexical templates.]
    The engine was constructed to interpret topologies through parsing of lexical files.
  \item[External library.]
    24 cloud providers are supported by the engine, through the advantage of 
    utilizing an external library.
  \item[Models@run.time.]
    \myac{M@RT} was utilized and combined with observer pattern to
    achieve asynchronous provisioning.
\end{description}

\mychapter{perspectives}{Perspectives}

\note{Full deployment is planed for next version of CloudML.}

\note{Live managing.}

\paragraph{Load balancer.}
\begin{figure}[tb]
  \begin{center}
    \begin{minted}[mathescape,
                   linenos,
                   numbersep=5pt,
                   frame=lines,
                   framesep=2mm]{json}
{
  "name": "test",
  "loadBalancer": {
    "name": "test",
    "protocol": "http",
    "loadBalancerPort": 80,
    "instancePort": 80
  }, 
  "nodes": []
}
    \end{minted}
  \end{center}
  \caption{Template including load balancer.}
  \label{list:loadbalancer}
\end{figure}




\glsaddall
\printglossaries

\appendix
%% Removed because of Copyright issues
%% \chapter{First appendix - Towards CloudML}
%% \includepdf[pages=-]{pdfs/cloudmde.pdf}
%% \chapter{Second appendix - Cloud-Computing}
%% \includepdf[pages=-]{pdfs/benevol.pdf}

\backmatter{}
\bibliography{bibliography,seb}
\bibliographystyle{plainnat}
\end{document}
